\section{Related Work}
%we need to talk about the newly fse lingming's paper, code not available, and also their results on benchmark is even lower than deepFL.


Lingming's paper GRACE has code now, but we won't have time to implement it and also it is designed for method ranking, our approach is proposed for statement-level. Their model is designed for method-level fault localization, aiming to provide rankings of faulty methods, which is different from our work.

also talk about the spectrum and mutation work, also the predicates-based approaches (icse'21 newly work).



 %has been intensively studied in the literature, e.g., Ochiai~\cite{abreu2006evaluation} and Jaccard~\cite{abreu2007accuracy}. 
Different types of techniques have been proposed, from Spectrum-based Fault Localization (SBFL)~\cite{zhang2011localizing, abreu2007accuracy, jones2005empirical, abreu2006evaluation, naish2011model, wong2007effective, liblit2005scalable, lucia2014extended}


Mutation-based Fault Localization (MBFL)~\cite{MUSE, zhang2013injecting, budd1981mutation, zhang2010test, musco2017large} 
aims to additionally consider impact information for fault localization, e.g., Metallaxis~\cite{papadakis2015metallaxis} and MUSE~\cite{MUSE}. 


to learning-to-rank~\cite{MULTRIC,TraPT,b2016learning,sohn2017fluccs} 


and casual inference techniques~\cite{kuccuk2021improving} and 



Neural networks~\cite{zheng2016fault, briand2007using, zhang2017deep, wong2009bp}.  However, they mainly work on the test coverage information, which has clear limitations (e.g., it cannot distinguish elements accidentally executed by failed tests and the actual faulty elements)~\cite{TraPT}. 


Some recent deep learning based approaches, such as GRACE~\cite{lou2021boosting}, and DeepFL~\cite{DeepFL}, can achieve better results than the above other types. 
