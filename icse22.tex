\PassOptionsToPackage{table,xcdraw}{xcolor}

\documentclass[sigconf,review,anonymous]{acmart}
%\acmConference[ESEC/FSE 2021]{The 29th ACM Joint European Software Engineering Conference and Symposium on the Foundations of Software Engineering}{23 - 27 August, 2021}{Athens, Greece}

\acmConference[ICSE 2022]{The 44th International Conference on Software Engineering}{May 21–29, 2022}{Pittsburgh, PA, USA}

%\documentclass[sigconf,review, anonymous]{acmart}
%\documentclass[sigconf]{acmart}

\usepackage{booktabs}   %% For formal tables:
                        %% http://ctan.org/pkg/booktabs
\usepackage{subcaption} %% For complex figures with subfigures/subcaptions
                        %% http://ctan.org/pkg/subcaption
\usepackage{array}
\usepackage{amsmath,amsfonts}
\usepackage{algorithm}
\usepackage[noend]{algpseudocode}
%\usepackage{algorithmic}
\usepackage{graphicx}
\usepackage{textcomp}
\usepackage{float} 
\usepackage{listings}
\usepackage{xspace}
\usepackage{multirow}
\usepackage{amsthm}
\newtheorem{definition}{Definition}
\usepackage{balance}

\usepackage[skins]{tcolorbox}

\usepackage{xcolor,pifont}
\newcommand*\colourcheck[1]{%
	\expandafter\newcommand\csname #1check\endcsname{\textcolor{#1}{\ding{52}}}%
}
\colourcheck{blue}
\colourcheck{green}
\colourcheck{red}

\newtcolorbox{myframe}[2][]{%
  enhanced,colback=white,colframe=black,coltitle=black,
  sharp corners,
  toprule=1.0pt,
  rightrule=0.3pt,
  leftrule=0pt,
  bottomrule=0pt,
  fonttitle=\itshape\scshape\large,
  left=0pt,right=5pt,top=5pt,bottom=3pt,
  attach boxed title to top right={yshift=-0.3\baselineskip-0.4pt,xshift=-5mm},
  boxed title style={tile,size=minimal,left=0.2mm,right=0.5mm,
    colback=white,before upper=\strut},
  title=#2,#1
}

%\newcommand{\code}[1]{{\footnotesize\textsf{#1}}}

\newcommand{\tool}{\textsc{FixLocator}\xspace}

\newtheorem{Definition}{Definition}
\newtheorem{Claim}{Claim}
\newtheorem{Lemma}{Lemma}
\newtheorem{Theorem}{Theorem}

\newcolumntype{L}[1]{>{\raggedright\arraybackslash}p{#1}}
\newtheorem{observation}{Observation}
\newtheorem{property}{Property}
\newcommand{\code}[1]{{\footnotesize\texttt{#1}}}
\usepackage{amsthm}
 \definecolor{dkgreen}{rgb}{0,0.6,0}
\definecolor{gray}{rgb}{0.5,0.5,0.5}
\definecolor{mauve}{rgb}{0.58,0,0.82}
\lstset{frame=tb,
  language=Java,
  aboveskip=3mm,
  belowskip=3mm,
  showstringspaces=false,
  columns=flexible,
  basicstyle={\small\ttfamily},
  numbers=left,
  numberstyle=\tiny\color{gray},
  keywordstyle=\color{blue},
  commentstyle=\color{dkgreen},
  stringstyle=\color{mauve},
  breaklines=true,
  breakatwhitespace=true,
  tabsize=4
}



\begin{document}

%\title[{\tool}: Deep Fault Localization with Code Coverage Representation Learning]{{\tool}: Deep Fault Localization with Code Coverage Representation Learning}

\title[{\tool}: Fault Localization to Detect Fixing Locations]{{\tool}: Fault Localization to Detect Fixing Locations}


%%%---- AUTHORS BLOCK ------

%Yi Li:New Jersey Institute of Technology;Shaohua Wang:New Jersey
%Institute of Technology;Tien Nguyen:University of Texas at Dallas 

%\author{Yi Li}
%\affiliation{
%\institution{New Jersey Inst. of Technology, USA}
%}
%\email{yl622@njit.edu}
%\author{Shaohua Wang}
%\affiliation{
%\institution{New Jersey Inst. of Technology, USA}
%}
%\email{davidsw@njit.edu}
%\author{Tien N. Nguyen}
%\affiliation{
%\institution{University of Texas at Dallas, USA}
%}
%\email{tien.n.nguyen@utdallas.edu}


%\renewcommand{\shortauthors}{Li, Wang, and Nguyen}

\setcopyright{none}

\settopmatter{printacmref=false, printfolios=false}

\renewcommand\footnotetextcopyrightpermission[1]{} % removes footnote with conference information in first column


%(1) present information sorted in a way that a CNN can "see" patterns
%discriminating between faulty and non faulty statements more easily;

%(2) identify the actual crash statement to the network;

%(3) present more information to the deep neural network in the form of
%a summary of data dependences for each statement as well as source
%embedding; and

%(4) the suspiciousness of a statement is seen taking into account
%relationships to other statement, as opposed to a statement by itself”



%\input{sections/abstract}
\begin{abstract}
We present {\tool}, a DL-based fault localization (FL) approach
supporting the detection of buggy statements in one or multiple
methods that need to be modified accordingly in the same fix. We treat
this FL as a dual learning task with two models. First, the
method-level pairing model, \code{MethFL}, learns the methods to be
fixed together. Second, the statement-level pairing
model, \code{StmtFL}, learns the statements to be co-fixed. Learning
for a model can benefit for the other and vice versa. Exploring this
duality provides useful constraints for {\tool} to learn derive
co-fixing statements. Thus, we joinly train them simultaneously with
soft-sharing the parameters of this dual models to exploit this
duality. In addition to the new dual learning model solution, we also
explore a novel feature, which is the co-change information among
statements. We use Graph-based Convolution Network to integrate
different types of program dependencies among the statements. Our
empirical evaluation on real-world datasets shows that {\tool}
relatively improves over the state-of-the-art statement-level FL
baselines from {\bf XXX\% to XXX\%}, and reduces the statements to be
examined {\bf XXX\%},
\end{abstract}


%\settopmatter{printacmref=true, printccs=true, printfolios=false}

%\begin{CCSXML}
%<ccs2012>
%<concept>
%<concept_id>10011007.10011006.10011073</concept_id>
%<concept_desc>Software and its engineering~Software maintenance tools</concept_desc>
%<concept_significance>500</concept_significance>
%</concept>
%</ccs2012>
%\end{CCSXML}

%\ccsdesc[500]{Software and its engineering~Software maintenance tools}

%\keywords{Deep Learning; Automated Program Repair; Context-based Code Transformation Learning}


\maketitle

\section{Introduction}

Detecting and fixing software defects is crucial for a software
development process. To reduce the efforts from developers in that
process, several {\em fault localization} (FL)
approaches~\cite{fl-survey} have been introduced to help localize the
source of the fault that needs to be fixed. The input of an FL model
is the execution of a test suite, in which some of the test cases are
passing or failing ones. Specifically, the key input is the {\em code
  coverage matrix} in which the rows and columns correspond to the
statements and test cases, respectively.  Each cell is assigned with
the value of 1 if the respective statement is executed in the
respective test case, and with the value of 0, otherwise.  An FL model
uses such information to identify the list of {\em suspicious lines of
  code} that are ranked based on their associated {\em suspiciousness
  scores}~\cite{fl-survey}. In recent advanced FL, several approaches
also support fault localization at method
level~\cite{DeepFL,icse21-fl}. 



%In the FL problem, given the execution of test cases, an FL tool
%identifies the set of {\em suspicious lines of code} with their
%associated suspiciousness scores~\cite{fl-survey}.  The key input of
%an FL tool is the {\em code coverage matrix} in which the rows and
%columns correspond to the source code statements and test cases,
%respectively.  Each cell is assigned with the value of 1 if the
%respective statement is executed in the respective test case, and with
%the value of 0, otherwise. In recent FL, several researchers also
%advocate for fault localization at method level~\cite{DeepFL}. FL at
%both levels are useful for developers.

The FL approaches can be broadly divided into the following
categories: {\em spectrum-based fault localization} (SBFL)
approaches~\cite{Ochiai,jones2001visualization,keller2017critical},
{\em mutation-based fault localization} (MBFL)
approaches~\cite{MUSE,papadakis2012using,Metallaxis}, and {\em machine
  learning (ML)} and {\em deep learning (DL)}~\cite{DeepFL,icse21-fl}.
For SBFL approaches, the key idea is that a line covered more in the
failing test cases than in the passing ones is more suspicious than a
line executed more in the passing ones.
%
To improve SBFL, MBFL
approaches~\cite{MUSE,papadakis2012using,Metallaxis} enhance the code
coverage information by modifying a statement with mutation operators,
and then collecting code coverages when executing the mutated programs
with the test cases. The MBFL approaches apply suspiciousness score
formulas in the same manner as in SBFL approaches on the matrix for
each original statement and its mutated code.
%
ML and DL-based FL approaches explore the code coverage matrix and
apply different neural network models for fault localization.


%{\em Spectrum-based fault localization} (SBFL)
%approaches~\cite{Ochiai,jones2001visualization,keller2017critical}
%take the recorded lines of code that were covered by each of the given
%test cases, and assigned each line of code a suspiciousness score
%based on the code coverage matrix. Despite using different
%formulas to compute that score, the idea is that a line covered more
%in the failing test cases than in the passing ones is more suspicious
%than a line executed more in the passing ones. A key drawback of those
%approaches is that the same score is given to the lines that have been
%executed in both failing and passing test cases. An example is the
%statements that are part of a block statement and executed at the
%same nested level. Another example is the conditions of the
%condition statements, e.g., \code{if}, \code{while}, \code{do},
%and \code{switch}. 

%To improve SBFL, {\em mutation-based fault localization} (MBFL)
%approaches~\cite{MUSE,papadakis2012using,Metallaxis}
%enhance the code coverage information by modifying a statement with
%mutation operators, and then collecting code coverages when executing
%the mutated programs with the test cases. They apply suspiciousness
%score formulas in the same manner as the spectrum-based FL approaches on
%the code coverage matrix for each original statement and its mutated
%ones. Despite the improvement, MBFL are not effective for the bugs
%that require the fixes that are more complex than a mutation
%(Section~\ref{motivexample}).

%{\em Machine learning (ML)} and {\em deep learning (DL)}
%have been used in fault localization. DeepFL~\cite{DeepFL}
%computes for each faulty method a vector with +200 scores in which
%each score is computed via a specific feature, e.g., a spectrum-based
%or mutation-based formula, or a code complexity metric. Despite its
%success, the accuracy of DeepFL is still limited. A reason could be
%that it uses various calculated scores from different formulas as a
%proxy to learn the suspiciousness of a faulty element, instead of
%fully exploiting the code coverage. Some formulas, such
%as the spectrum- and mutation-based formulas, inherently suffer from
%the issues as explained earlier with the statements covered by
%both failing and passing test cases.

Despite their successes, the state-of-the-art FL approaches still do
not support locating all dependent fixing locations that need to be repaired
at the same time in the same fix. In real-world software development,
there are several bugs that require a fix to multiple lines of code in
one or multiple hunks in the same or different methods. {\em The
  fixing changes to those lines of code are dependent to one another
  and need to be made in the same fix for the program to pass the test
  cases}. For those bugs, applying the fixing change to one statement
at a time will not make the program pass the test case after the
change to one statement.
%
This capability to detect the fixing locations of the co-changes in a
fix for a bug (let us call it {\em Co-change Fixing Locations} ({\em
  CC Fix Locations})) is important for both the manual process of bug
fixing as well as the automated process of program repair. For
manual process, such capability will save effort and time for
developers in locating all the buggy statements that need to be fixed
at the same time. For automated program repair (APR), such capability
will enable an APR model to correctly and completely make the changes
to fix a bug.

From the ranked list of suspicious statements returned from an
existing FL model, a naive approach to detect CC Fix Locations would
be to take the top $k$ statements in that list and to consider them as
to be fixed together. This solution might not be effective
because the mechanisms used in the state-of-the-art FL approaches have
never considered the co-change nature of those fixes. Our empirical
evaluation also confirmed that (Section~\ref{eval:sec}).

Detecting all the fixing locations at multiple statements in
potentially multiple methods is challenging. A naive solution would be
detecting the potential methods that need to be fixed together and
then detecting potential statements that need to be changed together
in each of those methods. However, doing so will create a comfounding
effect from the inaccuracy of the detection of buggy methods to the
detection of the buggy statements.

We propose {\tool}, a fault localization approach for buggy
statements/methods that can locate multiple buggy statements in
possible multiple buggy methods. To avoid the confounding effect in
the above naive solution, our solution treats this problem as a {\em
  dual learning} task with two models. First, the {\em method-level
  FL} (\code{MethFL}) model learns the methods that need to be
modified in the same fix. Second, the {\em statement-level FL}
(\code{StmtFL}) model learns the co-fixing statements in the same or
different methods. The intuition is that they are closely related,
which we refer to as {\em duality}. Learning for a model can benefit
for the other and vice versa. If two statements in two methods are
fixed together (co-fixed) for a bug, those methods are also
co-fixed. If two methods are co-fixed, some of their statements
are also co-fixed. Exploring this~duality can provide useful
constraints for our model to detect CC fixing locations. Thus,
instead of cascading them, we train them simultaneously with
soft-sharing the parameters of the models to exploit this
duality. We apply a probabilistic correlation as a regulization term
in the loss function in the join training. We also design a novel
constraint about the attention mechanism in the two models.

In addition to the new dual learning model solution, we also explore a
novel feature for this CC fixing location problem: co-change
statements, which are the ones changed together. The rationale to
consider general co-changes is that those co-changed statements in the
past might become the statements that will be fixed together in the
future. Finally, because the co-fixed statements are often dependent
to one another, we use Graph-based Convolution Network
(GCN)~\cite{li2019gcn} to integrate different types of program
dependencies among the statements including program dependence graph
(PDG), and execution path (EP). We also encode test coverage and the
co-change statements in the graph. The GCN model will learn and class
the bugginess of the statements.

%The statements that need to be fixed together are often dependent on
%one another with program dependencies. Thus, we use Graph-based
%Convolution Network (GCN)~\cite{li2019gcn} to model different types of
%dependencies among the statements. Thus, we encode both the program
%dependence graph (PDG) and execution path (EP) in our graph modeling.
%For convenience, we also encode the co-change relations among the
%statements into our general graph representations with different types
%of edges representing different relations. The GCN model enables both
%nodes’ and edges’ attributes and learns to classify the nodes.

{\bf FIXME.} We conducted several experiments to evaluate {\tool} on
Defects4J benchmark~\cite{defects4j}. Our empirical results show that
\tool locates 245 faults and 71 faults at the method level and the
statement level, respectively, using only top-1 candidate (i.e., the
first ranked element is faulty). It can improve the top-1 results of
the state-of-the-art \textit{statement-level} FL baselines by 317.7\%,
273.7\%, 173.1\%, 195.8\%, and 491.7\% when comparing with
Ochiai~\cite{Ochiai}, Dstar~\cite{DStar}, Muse~\cite{MUSE},
Metallaxis~\cite{Metallaxis}, and RBF-Neural-Network-based FL
(RBF)~\cite{RBF_Neural_Network}, respectively.  {\tool} also improves
the top-1 results of the existing \textit{method-level} FL baselines,
MULTRIC~\cite{MULTRIC}, FLUCCS~\cite{FLUCCS}, TraPT~\cite{TraPT}, and
DeepFL~\cite{DeepFL}, by 206.3\%, 53.1\%, 57.1\%, and 15.0\%,
respectively. Our results show that three sources of information in
{\tool} positively contribute to its high accuracy. We also evaluated
{\tool} on ManyBugs~\cite{LeGoues15tse}, a~ben\-chmark of C code with
9 projects. The results are~consistent with the ones on Java code.
{\tool} localizes 27 faulty statements and 98 faulty methods using
only top-1 results.

The contributions of this paper are listed as follows:

{\bf 1. {\tool}: Novel DL-based fault localization approach} that
derives the co-change fixing locations for a bug. Our idea is
to treat such problem as a dual learning task with the joint training
of the method-level and statement-level co-fixing learning models.

{\bf 2. Novel graph-based representation learning with co-change
  statements.} Our graph-based representation learning with GCN
and the novel type of features in co-change statements enables
the dual-task models learn derive co-change fixing locations.

{\bf 3. Extensive empirical evaluation.} We evaluated {\tool} against
the most recent FL models to show our model's better performance. Our
replication package is available at~\cite{FixLocator2022}.


\section{Motivating Example}
\label{motiv:section}

\begin{figure}[t]
	\centering
	\lstset{
		numbers=left,
		numberstyle= \tiny,
		keywordstyle= \color{blue!70},
		commentstyle= \color{red!50!green!50!blue!50},
		frame=shadowbox,
		rulesepcolor= \color{red!20!green!20!blue!20} ,
		xleftmargin=1.5em,xrightmargin=0em, aboveskip=1em,
		framexleftmargin=1.5em,
                numbersep= 5pt,
		language=Java,
    basicstyle=\scriptsize\ttfamily,
    numberstyle=\scriptsize\ttfamily,
    emphstyle=\bfseries,
                moredelim=**[is][\color{red}]{@}{@},
		escapeinside= {(*@}{@*)}
	}
	\begin{lstlisting}[]
public void toSource(final CodeBuilder cb, final int inputSeqNum, final Node root) {
	......
(*@{\color{red}{-	String code = toSource(root, sourceMap);}@*)
(*@{\color{cyan}{+	String code = toSource(root, sourceMap, inputSeqNum == 0);}@*)
	if (!code.isEmpty()) {
		cb.append(code);
	}
	......
}
//--------------------------------------------------------------------------
(*@@Override@*)
String toSource(Node n) {
	initCompilerOptionsIfTesting();
(*@{\color{red}{-	return toSource(n, null);}@*)
(*@{\color{cyan}{+ 	return toSource(n, null, true);}@*)
}
//--------------------------------------------------------------------------
(*@{\color{red}{- private String toSource(Node n, SourceMap sourceMap) {}@*)
(*@{\color{cyan}{+ private String toSource(Node n, SourceMap sourceMap, boolean firstOutput) {}@*)
	......
  builder.setSourceMapDetailLevel(options.sourceMapDetailLevel);
(*@{\color{red}{-   builder.setTagAsStrict(}@*)
(*@{\color{cyan}{+   builder.setTagAsStrict(firstOutput}@*) && 
		options.getLanguageOut(a) == LanguageMode.ECMASCRIPT5_STRICT);
  builder.setLineLengthThreshold(options.lineLengthThreshold);
	......	
}	
	\end{lstlisting}
        \caption{A Multi-statement/Multi-method Bug Fix}
         \label{fig:motiv}
\end{figure}

Let us start with a real-world example on the kind of bug fixes that
require multiple changes to multiple statements of different methods.
Figure~\ref{fig:motiv} shows a bug in Defects4J dataset. The bug
occurred when the method call to \code{setTagAsStrict} did not
consider the first output in its arguments. Therefore, to fix it, a
developer adds a new argument in the method \code{toSource} at line
19, and use that argument in the method call \code{setTagAsStrict
  (firstOutput,...)} at line 23. Because the method \code{toSource} at
line 19 was changed, the two callers at line 3 of the method
\code{toSource} (line 1) and at line 14 of the method \code{toSource}
(line 12) need to be changed accordingly.

%This is a bug from Defects4J data that has three methods need to be fixed at the same time. The bug is caused by doing the method call $setTagAsStrict$ without checking if it is the first output. To fix this bug, firstly, for the method C, we need to add one more parameter $firstOutput$ for the method call $setTagAsStrict$. Then, for the method A and B, because they call the method C or override the method C, when adding a new parameter in the method C, they all need to do the same thing. We need to add $inputSeqNum == 0$ in method A and add $true$ in method B when calling the method C.

%In this bug, there are multiple statements that need to be fixed and they located in three different methods. Also, when analyzing the co-change information, the method A and method C have been fixed together before in the commit history. 

\subsection{Key Ideas}
\label{sec:key-ideas}

To address those challenges, we propose {\tool}, a novel FL approach
to locate all the CC fixing locations (i.e., faulty statements) that need
to be modified in a fix for a given bug. {\tool} has the following
novel key ideas in both new model and features:

{\bf Key Idea 1. [Dual Learning for Fault Localization]} To avoid the
confounding effect in a naive solution of detecting faulty methods
first and then detecting faulty statements in those methods, we design
an approach that treats detecting dependent CC fixing locations as a
{\em dual learning} task between them. First, the {\em method-level
  FL} model (\code{MethFL}) aims to learn the methods that need
to be modified in the same fix. Second, the {\em statement-level
  FL} model (\code{StmtFL}) aims to learn the co-fixing
statements regardless of whether they are in the same or different
methods.

Intuitively, \code{MethFL} and \code{StmtFL} are related to each
other, in which the results of one model can help the other. We refer
to this relation as {\em duality}, which can provide some useful
constraints for {\tool} to learn dependent CC fixing locations.
%
We conjecture that the joint training of the two models can improve
the performance of both models, when we leverage the constraints of
this duality in term of shared representations. For example, if two
statements in two different methods $m_1$ and $m_2$ were observed to
be changed in the same fix, then it should help the model learn that
$m_1$ and $m_2$ were also changed together to fix the bug.  If two
methods were observed to be fixed together, then some of their
statements were changed in the same fix as well. In our model, we
jointly train \code{MethFL} and \code{StmtFL} with the models'
parameter soft-sharing to exploit their relation. We use a mechanism
called {\em cross-stitch unit} to learn a linear combination of the
input features from two models to {\em enable the propagation of the impact
of \code{MethFL} on \code{StmtFL} and vice versa}.

%We also add an attention mechanism in the two models to help emphasize
%on the key features.


%Therefore, the third component is the {\em localization model} with
%the main goal of deriving the CC fixing locations. Another task of
%this component is to train the MethFL and StmtFL simultaneously to
%explot the duality. Specifically, we apply a probabilistic correlation
%as a regualization term in the foss function in the join training. We
%also design a novel constraint about the attention mechanism in two
%models.

%two dual regularization terms to constrain the duality of the two
%models, which are enlightened by the probabilistic correlation and the
%symmetry of attention weights between two models.




%{\bf Key Idea 3. [Multi-task Learning Fault Localization]} When doing the fault localization, there are often two levels of fault localization that we can do: the statement-level fault localization and the method-level fault localization. More detailed fault localization can help the developers to find the bugs easier. So for\tool, we regard the statement-level fault localization as our primary goal. However, the method fault localization is still helpful because the method-level fault localization often has higher accuracy and can help reduce the biases. 

%To make the method-level fault localization helpful when doing the statement-level fault localization, in \tool, we build a multi-task learning framework to do the statement-level fault localization. To be more specific, we have two separate models for the two levels of fault localization. The first model is to do the statement-level fault localization as we mentioned in key idea 2. And the second model is to do the method-level fault localization. In this model, we take the method pairs as input and train the model to predict if the method pairs are co-changes or not. During the training process, the parameters between the first and second models are softly shared to build the multi-task learning framework. We will introduce the details in the approach section.

{\bf Key Idea 2. [Co-Change Representation Learning]} In our problem
of detecting CC fixing locations, in addition to a new dual-learning
model in key idea 1, we also use a new feature: {\em co-change
  information}, i.e., the statements/methods that were changed
together. The co-changed statements/methods in the same commit are
used to train the two models. We also consider the co-fixed
statements/methods in the same fixes. The rationale for considering
general co-changes because those co-changed entities in the past might
become the ones that will be fixed together in the future.

%Just as we mentioned in observation 1, co-change is very common when
%we want to fix a bug or do an enhancement on the code. To catch the
%co-change information, we first collect the commits that change the
%source code. And for each commit, we mark the statements that changed
%together as the co-change. By collecting all co-changes from the
%commits in the project, we have a large co-change history
%dataset. Then, to analyze the code relationship, we build a link
%between the statements been changed together. To make the co-change
%information useful, we add these edges into the other graph to do the
%graph modeling. That is our second key idea.

{\bf Key Idea 3. [Graph Modeling for Dependencies among Statements/Methods]} The
statements/methods that need to be fixed together are
interdependent with several dependencies. Thus, we
use Graph-based Convolution Network (GCN)~\cite{li2019gcn} to model
different types of dependencies among the statements and methods,
e.g., data and control dependencies in a program dependence graph
(PDG), execution traces, stack traces, etc. For convenience, we also
encode the co-change relations among the statements into our graph
representations with different types of edges representing different
relations. The GCN model enables both nodes' and edges' attributes and
learns to classify the nodes as buggy or~not.

%{\bf Key Idea 2. [Multi-edge-types Graph Modeling]}

%As mentioned in observation 2, one bug can be involved in multiple
%methods, and for each method, there can be more than one statement
%related to the bug. To analysis the relationships between the
%statements that in the same method, the graph modeling between the
%source code such as program dependency graph (PDG), execution paths
%(EP), and co-change relationships (CCR) is the way that we thought to
%be suitable. To avoid overlapping, we regard the PDG as the based
%graph, add the EP and CCR edges into the graph, and use the combined
%graph to analyze the relationship.

%However, the different types of edges in one graph cannot be regarded as the same type. To specify the differences among the four different types of edges (PDG contains data dependency and control dependency, two types of edges), we use an advanced GCN \cite{li2019gcn} that takes both node and edge attributes as inputs to learn the node classification. Because it accepts edge attributes, we give the different types of edges with different labels as input to specify their differences.







\section{Approach Overview}
\label{overview:sec}

\begin{figure*}[t]
	\centering
	\includegraphics[width=5.6in]{graphs/overview.png}
	\caption{{\tool}: Training Process}
	\label{train-overview}
\end{figure*}

{\tool} has two main processes. Figure~\ref{train-overview} displays
the general architecture of the training process. The input of the
training process is the passing and failing test cases, as well as the
source code repository of the project under study. The output is the
parameters for the method-level pairing model (detecting co-fixed
methods) and the statement-level pairing model (detecting co-fixed
statements). The training process has three main steps:

\subsubsection*{\underline{Step 1. Feature Extraction}}
The goal of the first step is to extract the important features for FL
from the test coverage and source code (including co-changes). The
features are extracted from two levels: statements and methods. At
each level, we extract the important {\em attributes} of statements or
methods, as well as the crucial {\em relations} among them. Thus, we
use graphs to represent those attributes and relations. Let us call
them the graph-based features.

For \underline{a method $m$}, we collect as its attributes 1) method
content: the sequences of the sub-tokens of its code tokens (excluding
separators and special tokens), and 2) method structure: the Abstract
Syntax Tree (AST) of the method. For the relations among methods, we
extract the relations involving in

1) execution flow (the calling relation, i.e., $m$ calls $n$),

2) stack trace after a crash (the order relation among the methods in
the stack trace) (the dynamic information in execution paths and stack
traces has been showed to be useful in FL~\cite{icse21-fl,DeepFL}),

3) co-change relation in the project history (two methods that were
changed in the same commit are considered to have the co-change
relation),

4) similarity: we also extract the similar methods in the project that
have been buggy before in the project history. We keep only the most
similar method for each method (the similarity is measured by the one
from two sequences of sub-tokens in the two methods).

For \underline{a statement $s$}, we extract both static and dynamic
information. First, for static information, we extract the subtree in
the AST that corresponds to the statement to represent its
structure. We also extract the list of variables in the statement $s$
together with its type, forming a sequence of names and types. Second,
for the dynamic information, we encode the test coverage matrix for
$s$ as two vectors. In the first vector, the element corresponding to
the test case $i$ will be 1 if the test case covers the $s$ and 0
otherwise. In the second vector, the element for the test case $i$
will be 1 if it is passing and 0 otherwise.

\subsubsection*{\underline{Step 2. Graph-based Feature Representation Learning}}

\section{Feature Extraction}

The first step of \tool is to preprocess the input data into the suitable format and group them into statement and method two levels. So the input of this step is the \tool input, including the java project that needs to do the fault localization with the commit history and the relevant test cases for the project. And the output of this step is two groups of graphs with node features for statement-level and method-level.

Specifically, \tool extract the features from two levels: the method-level and statement-level feature extraction.

\begin{figure}[t]
	\centering
	\includegraphics[width=3in]{graphs/step-1-method.png}
	\caption{Method-level Feature Extraction}
	\label{method-level-feature-extraction}
\end{figure}

\subsection{Method-level Feature Extraction}
For the method level, \tool uses the method as the basic unit. And for each method, \tool extracts three key features, including the method content, abstract syntax tree (AST), and the most similar buggy method. 1> For the method content, \tool collect the source code of each method $M$ and link each statement $S_m$ one by one in $M$ as a sequence $Seq_m$ to represent the method content. \tool removes all special characters and uses CamelCase to break down the tokens in the sequence into sub-tokens to reduce the influence of biases. For example, in the Figure \ref{method-level-feature-extraction}, the method content feature in the method $M1$ shows the extracted sequence of sub-tokens $protected void compute ......$. The source code is listed on the input of the Figure \ref{statement-level-feature-extraction}. 2> As for the AST, \tool generates abstract syntax tree $Tree_m$ for each method $M$ by using JDT package \cite{JDT}. The generated AST looks like the example shown on the abstract syntax tree feature in Figure \ref{method-level-feature-extraction}. 3> When extracting the most similar buggy method feature, \tool breaks down the methods into the sequence of sub-tokens just like the method content feature. It then uses GloVe \cite{pennington2014glove} to learn the embedding for each sub-token and replace the sub-tokens with the embedding vectors. After this, \tool calculates the cosine similarity between the current method $m$ and all other buggy methods in the commit history (before the current bug) to find the most similar buggy method $m_b$. For this buggy method, \tool also uses the JDT package to generate the AST for $m_b$. The most similar buggy method feature in Figure \ref{method-level-feature-extraction} shows a possible example for it.

After having these three features for each method, \tool uses three kinds of edges to link them as a graph. First of all, \tool runs test cases for the project. If a test case $t_i$ failed, \tool collect the stack trace for the test case $t_i$. Because the stack trace is sometimes too long, using the crashed position as the root, \tool only picks part of the stack trace $st_i$ with the depth of ten in consideration. It means that on the stack trace $st_i$, there are at most ten methods include $m_1, m_2, ..., m_{10}$ where $m_1$ is the crashed position. For every two method $m_j$ and $m_{j+1}$ among these methods, $m_{j+1}$ calls the $m_j$ in the stack trace. The edge $E_m^s$ direction in it is always from $m_{j+1}$ point to $m_j$. For example, in the Figure \ref{method-level-feature-extraction}, the crashed method is $M1$ and the blue solid edges are the stack trace $st_i$. The figure shows that in the stack trace, $M3$ comes out before $M2$ and $M2$ comes out before $M1$. The second type of edge is the execution path. By using each method $m_j$ in the stack trace $st_i$ as the root method, \tool expands the stack trace $st_i$ by adding the executed methods into the graph. The direction of the execution edges $E_m^e$ is also the same as the call direction. Also, because sometimes the execution path may be very long for a method, we only keep the methods $m_k$ within ten steps from the crashed position $m_1$ that means in the graph, from node $m_k$ to $m_1$, the steps are no more than ten (when counting the steps, \tool ignore the edge direction). The green dotted line in Figure \ref{method-level-feature-extraction} shows an example for this type of edge. With the Figure \ref{method-level-feature-extraction}, the green dotted line reflects the relationship that when running the test case $t_i$ on $M2$, it also executes the method $M4$ based on the method call. And then, it executing the $M3$, it executes the method $M5$, and within $M5$, it also executes method $M6$ based on the method calls inside the methods. The third type of edge is the co-change relation. For this,  \tool collects all commit history from the input java project. If more than one method has changed in one commit, we mark it as a co-change. The co-change contains all the methods that changed together in this commit. To add the co-change relation as one type of edge, \tool makes the co-change relation become a two-directional edge $E_m^c$ (e.g. The orange edge between $M5->M6$ and $M6->M5$ in the Figure \ref{method-level-feature-extraction}).
%
%\begin{itemize}%
%	\item Graph: 
%	\begin{itemize}
%		\item Stack Trace: \tool runs test cases for the project. If a test case $t_i$ failed, \tool collect the stack trace for the test case $t_i$. Because the stack trace may be very bug, by using the crashed position as the root, \tool pick part of the stack trace $st_i$ with the depth of ten. It means that on the stack trace $st_i$, there are ten methods include $m_1, m_2, ..., m_{10}$ where $m_1$ is the crashed position and for every two method $m_j$ and $m_{j+1}$ among them, $m_{j+1}$ calls the $m_j$ in the stack trace. The edge $E_m^s$ direction in it is always from $m_{j+1}$ point to $m_j$ \tool uses $st_i$ as the base graph and the relationship information in $st_i$ is the dynamic information. 
%		\item Execution Path: As for the failed test case $t_i$, \tool also analyzes the execution path. By using each method $m_j$ in the stack trace $st_i$ as the root method, \tool expands the stack trace $st_i$ by adding the executed methods into the graph. The direction of the execution edges $E_m^e$ is also the same as the call direction. Also, because sometimes the execution path may be very long for a method, we only keep the methods $m_k$ within ten steps from the crashed position $m_1$ that means in the graph, from node $m_k$ to $m_1$, the steps are no more than ten (when counting the steps, \tool ignore the edge direction). The added execution information here is the dynamic edge.
%		\item Co-change Information: \tool collects all commit history of the input java project. If more than one java method has changed in one commit, we mark it as a co-change. The co-change contains all the methods that changed together in this commit. Because the co-change does not have the direction, there is one non-directional edge $E^c$ between every two methods in this co-change to represent the co-change relationship. In order to add the co-change relationship into the stack trace, \tool makes the non-directional edge $E^c$ become a two-directional edge $E_m^c$ (e.g.$method_A -> method_B$ and $method_B -> method_A$). This type of edge is the static edge.
%	\end{itemize}
%	\item Node Features: 
%	\begin{itemize}
%		\item Method Content: \tool collect the source code of each method $M$ and link each statement $S_m$ one by one in $M$ as a sequence $Seq_m$ to represent the method content. \tool removes all special characters and uses CamelCase to break down the tokens in the sequence into sub-tokens to reduce the influence of biases. For example, the $setTagAsStrict$ can be break down into $set, Tag, As,$ and $Strict$. The processed sequence of sub-tokens $Seq^p_m$ is used to represent the method content in \tool. This feature is one of the static features that \tool collects from the source code to represent the method.
%		\item Method Structure: \tool generates abstract syntax tree $Tree_m$ for each method $M$ by using JDT package \cite{JDT}. Each tree $Tree_m$ represent the structure of the relevant method $m$. This is one of the static feature that \tool collect from the source code to represent the method.
		%\item Similar Buggy Method: \tool breaks down the methods into the sequence of sub-tokens just like method content feature and then uses GloVe \cite{pennington2014glove} to learn the embedding for each sub-token and replace the sub-tokens with the embedding vectors. After this, \tool calculate the cosine similarity between the current method $m$ and all other buggy methods in the commit history (before current bug) to find the most similar buggy method $m_b$. This is one of the static feature that \tool collect from the source code to represent the method.
%	\end{itemize}	
%\end{itemize} 


\begin{figure}[t]
	\centering
	\includegraphics[width=3.4in]{graphs/step-1-statement.png}
	\caption{Statement-level Feature Extraction}
	\label{statement-level-feature-extraction}
\end{figure}

\subsection{Statement-level Feature Extraction}

For the statement level,  \tool uses the statement as the basic unit. And for each statement, \tool extracts three key features, including the code coverage information, sub-AST, and variables to represent the statement. 1> As for the code coverage information, \tool runs the relevant test cases for the input project. For the test case $t_i$, if it passes the statement $s_i$, \tool uses $c_i = 1$ to represent it while if it does not pass the statement $s_i$, \tool uses $c_i = 0$ to represent it. By linking all $c_i$ together, \tool can get $C = <c_1, c_2, ..., c_i>$. Also, \tool uses $r_i = 1$ to represent the the condition $passed$ and uses $r_i = 0$ to represent the condition $failed$ for the test case $t_i$ . By linking all $r_i$ together, \tool can get $R = <r_1, r_2, ..., r_i>$. By concatenate $C$ and $R$, \tool extract the code coverage information feature by $V_{coverage} = <c_1, c_2, ..., c_i, r_1, r_2, ..., r_i>$. The code coverage information feature in Figure \ref{statement-level-feature-extraction} shows an example of it. 2> For the sub-AST, it is very similar to the method level. By using JDT package, \tool can extract the AST for the whole method. And then, \tool searches for the nodes that appears in the statement. By collecting all these nodes and the edges between then, \tool can extract the sub-AST for the statement. 3> For the variables feature, for each statement, \tool collects all the variables $V$ that appeared in it and for each variable $v$ in $V$, \tool uses the $(variable_name variable_type)$ to represent it. Then \tool links all variables $V$ together with $,$ as a sequence $Seq_s$ as one of the static feature that \tool collect from the source code to represent the statement. For example, in Figure \ref{statement-level-feature-extraction}, \tool goes through the statement at $line 5$ in the input method and finds that only one variable $tree$ appears in the statement at $line 5$. So the variable feature here should be like $variable_name variable_type$ where $variable_name$ is $tree$ and $variable_type$ is $BSPTree Euclidean2D$.

With these three features, similar to the method-level, \tool builds three types of edges to link the statements together as a graph. The first type of edge is the program dependency edge (PD). \tool builds the PD by using the tool soot \cite{soot} for the method $m$. For example, in Figure \ref{statement-level-feature-extraction}, the blue edges are the PD edges and they show that in method $m$, statement at $line 4$ controls the statement at $line 5$. And statement at $line 5$ can control the statements at $line 7-8$ and the statements at $line 10-11$. The second type of edge that \tool extracts is the execution flow $E_s^e$. The execution flow is the order that the failed test case $t_i$ went through in the method $m$. For example, the green dotted edges are the execution flow. Within it, we can see that the test case executes $S7-S8$ but did not go through $S10-S11$. Even though in the PD, $S5$ is linked by $S10-S11$. It means that when running the test case $t_i$, it will not pass the $S10$ and $S11$ because of the if checking in $S5$. The last type of edge $E_s^c$ is the co-change relation in the statement level. \tool collects the co-change information for the commit about the statements that changed together before in the current method $m$ and one commit. In Figure \ref{statement-level-feature-extraction}, the commit history shows that $S4$ and $S5$ used to be changed together before. Hence, there is an orange edge that represents the co-change relation between $S4$ and $S5$.


%\begin{itemize}
%	\item Graph: 
%	\begin{itemize}
	%	\item Program Dependency Graph (PDG): \tool builds the PDG by using the tool soot \cite{soot} for the method $m$ that contains the statements that \tool want to analyze. \tool uses the generated PDG as the base graph. Within this graph, there are two types of edges including data dependency and control dependency. Both of these two types of edges are the static edges.
	%	\item Execution Path: \tool collects the execution path of the failed test case $t_i$ within the method $m$ and adds them into the PDG by adding a new type of edge $E_s^e$. The new edge direction is the same as the execution order. This type of edge is the dynamic edge.
%		\item Co-change Information: Similar to the method-level, \tool collects the co-change information for the commit about the statements that changed together before in one commit and the current method $m$. As for adding the co-change information into the PDG, \tool also creates the two-directional edge $E_s^c$ similar to the method level. This type of edge is the static edge.
%	\end{itemize}
%	\item Node Features: 
%	\begin{itemize}
	%	\item Code Coverage Information: \tool runs the relevant test cases for the input java project. For the test case $t_i$, if it passes the statement $s_i$, \tool uses $c_i = 1$ to represent it while if it does not pass the statement $s_i$, \tool uses $c_i = 0$ to represent it. By linking all $c_i$ together as $C = <c_1, c_2, ..., c_i>$, $C$ is considered by \tool as one of the dynamic feature.
	%	\item Statement Structure: Similar to the method-level, \tool generates a sub abstract syntax tree $Tree_s$ for each statement $S$ by using JDT \cite{} package. Each tree $Tree_s$ represent the structure of the relevant statement $s$. This feature is one of the static features that \tool collects from the source code to represent the statement.
	%	\item Variables: For each statement, \tool collects all the variables $V$ that appeared in it and for each variable $v$ in $V$, \tool uses the $(variable_name variable_type)$ to represent it. Then \tool links all variables $V$ together with $,$ as a sequence $Seq_s$ as one of the static feature that \tool collect from the source code to represent the statement. For example, the variables in the statement in line 3 in Figure \ref{fig:motiv} include $root$ and $sourceMap$. \tool generates the feature for them as $root Node, sourceMap SourceMap$ where $root$ and $sourceMap$ are the names and $Node$ and $SourceMap$ are the types.
%	\end{itemize}	
%\end{itemize} 

\section{Approach}
\subsection{Approach Overview}

In \tool, we have three main steps to do the fault localization. We want to introduce them as follow:

{\bf Step 1. Data Preprocess.} \tool accepts the java project with all the related test cases as the input for the whole tool. In this step, \tool firstly preprocesses the input into features and graphs in three main directions. And then \tool groups these features and graphs into two big sets for statement-level and method-level fault localization. \tool generates features include: 1) Method content (sequence of sub-tokens); 2) Abstract Syntax Tree (whole method); 3) Most similar buggy method; 4) Code coverage information; 5) Sub-AST (single statement); 6) Variables. \tool generates graphs include: 1) Stack traces; 2) execution path (method level); 3) execution path (statement level); 4) Co-change relationship (method level); 5) Co-change relationship (statement level); 6) Program Dependency Graph (PDG). After having these features, we group the feature 1), 2), and 3) with graphs 1), 2), and 4) as one set for the method-level fault localization usage while we group the rest features and graphs as one set for the statement-level fault localization.

{\bf Step 2. Feature Representation Learning} In this step, \tool uses different technologies to combine and learn the representations from the features and graphs generated in the last step. 

For the method-level, \tool collects the stack traces with the depth of 10 with directional edges $E_m^s$ and expands it by adding the execution path and co-change relationships. As for the execution path, \tool starts to find them from the stack traces with the depth of 10 and collects all the methods within ten steps from the crash position in the stack trace. \tool adds these methods by adding a new type of directional edge $E_m^e$ to the stack trace. As for the co-change relationships, \tool collects all co-change information from the repository commit history and adds them to the stack trace with the new type of non-directional edges $E^c$. If a set of methods $M$ has been changed together in previous commits in the repository, \tool regards them as the co-change information. There will be one non-directional edge $E^c$ between every two methods in this set. Because the stack trace graph is a directional graph, to add them to the stack trace graph, we make the non-directional edge $E^c$ become a two-directional edge $E_m^c$ (e.g.$method_A -> method_B$ and $method_B -> method_A$). After having the combined graph $G_m$, for each node, \tool uses RNN model \cite{} to learn the representation of the method content and uses the tree-based model TreeCaps \cite{} to learn the representation of the AST and the most similar buggy method. 

Similar to the method-level, on the statement-level, \tool uses TreeCaps to learn the representation for the sub-AST as one of the node features. And \tool uses the PDG as the base graph and adds the co-change relationship and execution path to build the combined graph $G_s$ in the same way as the method-level. The different features are the code coverage information and variables. For the code coverage information, \tool collects the test cases running results for a statement. If the test case $t_i$ passes the statement, \tool uses $1$ to represent the code coverage information $c_i$ for the test case $t_i$ while \tool uses $0$ to represent $c_i$ when the test case $t_i$ does not pass the statement. By collecting them together, \tool has the code coverage information feature as $<c_1, c_2, ..., c_i>$. And for the variables, \tool collects all variables $v$ that appear in the statement, and for each variable, \tool uses the variable name and type to represent it. By linking these variables as a sequence, \tool uses the RNN model to learn the representation for the variable feature.

{\bf Step 3. Multi-tasking Fault Localization} After having the feature representations from the last step, \tool uses a multi-tasking framework to do the fault localization. In the framework, there are two tasks: method-level fault localization and method-level fault localization. Among them, statement-level fault localization is the primary goal for the multi-tasking framework.

For the method-level fault localization, \tool uses the GCN model \cite{} to do the binary classification $C_m$ for each node based on the combined graph $G_m$. If the method is faulty, the GCN model will classify the node as $1$ while the GCN model classifies the node to $0$ when the method is non-faulty. Like the statement level, \tool uses the other GCN model to classify $C_s$ on the statement level based on the combined graph $G_s$.

When doing the training, \tool learns the classification on both method-level and state-level and does the soft parameter sharing between the two models to make these two GCN models learn more features from both levels. But when making the prediction, because the statement-level fault localization is the primary goal, \tool only picks the statement-level fault localization results as the final results. Thus, \tool regards all the statements marked as faulty as the model output statement set, which contains the statements predicted to be fixed together in this fault.

\begin{figure*}[t]
	\centering
	\includegraphics[width=6.5in]{graphs/overview.png}
	\caption{{\tool}'s Architecture}
	\label{overview}
\end{figure*}

\section{Feature Extraction}

The first step of \tool is to preprocess the input data into the suitable format and group them into statement and method two levels. So the input of this step is the \tool input, including the java project that needs to do the fault localization with the commit history and the relevant test cases for the project. And the output of this step is two groups of graphs with node features for statement-level and method-level.

Specifically, \tool extract the features from two levels: the method-level and statement-level feature extraction.

\begin{figure}[t]
	\centering
	\includegraphics[width=3in]{graphs/step-1-method.png}
	\caption{Method-level Feature Extraction}
	\label{method-level-feature-extraction}
\end{figure}

\subsection{Method-level Feature Extraction}
For the method level, \tool uses the method as the basic unit. And for each method, \tool extracts three key features, including the method content, abstract syntax tree (AST), and the most similar buggy method. 1> For the method content, \tool collect the source code of each method $M$ and link each statement $S_m$ one by one in $M$ as a sequence $Seq_m$ to represent the method content. \tool removes all special characters and uses CamelCase to break down the tokens in the sequence into sub-tokens to reduce the influence of biases. For example, in the Figure \ref{method-level-feature-extraction}, the method content feature in the method $M1$ shows the extracted sequence of sub-tokens $protected void compute ......$. The source code is listed on the input of the Figure \ref{statement-level-feature-extraction}. 2> As for the AST, \tool generates abstract syntax tree $Tree_m$ for each method $M$ by using JDT package \cite{JDT}. The generated AST looks like the example shown on the abstract syntax tree feature in Figure \ref{method-level-feature-extraction}. 3> When extracting the most similar buggy method feature, \tool breaks down the methods into the sequence of sub-tokens just like the method content feature. It then uses GloVe \cite{pennington2014glove} to learn the embedding for each sub-token and replace the sub-tokens with the embedding vectors. After this, \tool calculates the cosine similarity between the current method $m$ and all other buggy methods in the commit history (before the current bug) to find the most similar buggy method $m_b$. For this buggy method, \tool also uses the JDT package to generate the AST for $m_b$. The most similar buggy method feature in Figure \ref{method-level-feature-extraction} shows a possible example for it.

After having these three features for each method, \tool uses three kinds of edges to link them as a graph. First of all, \tool runs test cases for the project. If a test case $t_i$ failed, \tool collect the stack trace for the test case $t_i$. Because the stack trace is sometimes too long, using the crashed position as the root, \tool only picks part of the stack trace $st_i$ with the depth of ten in consideration. It means that on the stack trace $st_i$, there are at most ten methods include $m_1, m_2, ..., m_{10}$ where $m_1$ is the crashed position. For every two method $m_j$ and $m_{j+1}$ among these methods, $m_{j+1}$ calls the $m_j$ in the stack trace. The edge $E_m^s$ direction in it is always from $m_{j+1}$ point to $m_j$. For example, in the Figure \ref{method-level-feature-extraction}, the crashed method is $M1$ and the blue solid edges are the stack trace $st_i$. The figure shows that in the stack trace, $M3$ comes out before $M2$ and $M2$ comes out before $M1$. The second type of edge is the execution path. By using each method $m_j$ in the stack trace $st_i$ as the root method, \tool expands the stack trace $st_i$ by adding the executed methods into the graph. The direction of the execution edges $E_m^e$ is also the same as the call direction. Also, because sometimes the execution path may be very long for a method, we only keep the methods $m_k$ within ten steps from the crashed position $m_1$ that means in the graph, from node $m_k$ to $m_1$, the steps are no more than ten (when counting the steps, \tool ignore the edge direction). The green dotted line in Figure \ref{method-level-feature-extraction} shows an example for this type of edge. With the Figure \ref{method-level-feature-extraction}, the green dotted line reflects the relationship that when running the test case $t_i$ on $M2$, it also executes the method $M4$ based on the method call. And then, it executing the $M3$, it executes the method $M5$, and within $M5$, it also executes method $M6$ based on the method calls inside the methods. The third type of edge is the co-change relation. For this,  \tool collects all commit history from the input java project. If more than one method has changed in one commit, we mark it as a co-change. The co-change contains all the methods that changed together in this commit. To add the co-change relation as one type of edge, \tool makes the co-change relation become a two-directional edge $E_m^c$ (e.g. The orange edge between $M5->M6$ and $M6->M5$ in the Figure \ref{method-level-feature-extraction}).
%
%\begin{itemize}%
%	\item Graph: 
%	\begin{itemize}
%		\item Stack Trace: \tool runs test cases for the project. If a test case $t_i$ failed, \tool collect the stack trace for the test case $t_i$. Because the stack trace may be very bug, by using the crashed position as the root, \tool pick part of the stack trace $st_i$ with the depth of ten. It means that on the stack trace $st_i$, there are ten methods include $m_1, m_2, ..., m_{10}$ where $m_1$ is the crashed position and for every two method $m_j$ and $m_{j+1}$ among them, $m_{j+1}$ calls the $m_j$ in the stack trace. The edge $E_m^s$ direction in it is always from $m_{j+1}$ point to $m_j$ \tool uses $st_i$ as the base graph and the relationship information in $st_i$ is the dynamic information. 
%		\item Execution Path: As for the failed test case $t_i$, \tool also analyzes the execution path. By using each method $m_j$ in the stack trace $st_i$ as the root method, \tool expands the stack trace $st_i$ by adding the executed methods into the graph. The direction of the execution edges $E_m^e$ is also the same as the call direction. Also, because sometimes the execution path may be very long for a method, we only keep the methods $m_k$ within ten steps from the crashed position $m_1$ that means in the graph, from node $m_k$ to $m_1$, the steps are no more than ten (when counting the steps, \tool ignore the edge direction). The added execution information here is the dynamic edge.
%		\item Co-change Information: \tool collects all commit history of the input java project. If more than one java method has changed in one commit, we mark it as a co-change. The co-change contains all the methods that changed together in this commit. Because the co-change does not have the direction, there is one non-directional edge $E^c$ between every two methods in this co-change to represent the co-change relationship. In order to add the co-change relationship into the stack trace, \tool makes the non-directional edge $E^c$ become a two-directional edge $E_m^c$ (e.g.$method_A -> method_B$ and $method_B -> method_A$). This type of edge is the static edge.
%	\end{itemize}
%	\item Node Features: 
%	\begin{itemize}
%		\item Method Content: \tool collect the source code of each method $M$ and link each statement $S_m$ one by one in $M$ as a sequence $Seq_m$ to represent the method content. \tool removes all special characters and uses CamelCase to break down the tokens in the sequence into sub-tokens to reduce the influence of biases. For example, the $setTagAsStrict$ can be break down into $set, Tag, As,$ and $Strict$. The processed sequence of sub-tokens $Seq^p_m$ is used to represent the method content in \tool. This feature is one of the static features that \tool collects from the source code to represent the method.
%		\item Method Structure: \tool generates abstract syntax tree $Tree_m$ for each method $M$ by using JDT package \cite{JDT}. Each tree $Tree_m$ represent the structure of the relevant method $m$. This is one of the static feature that \tool collect from the source code to represent the method.
		%\item Similar Buggy Method: \tool breaks down the methods into the sequence of sub-tokens just like method content feature and then uses GloVe \cite{pennington2014glove} to learn the embedding for each sub-token and replace the sub-tokens with the embedding vectors. After this, \tool calculate the cosine similarity between the current method $m$ and all other buggy methods in the commit history (before current bug) to find the most similar buggy method $m_b$. This is one of the static feature that \tool collect from the source code to represent the method.
%	\end{itemize}	
%\end{itemize} 


\begin{figure}[t]
	\centering
	\includegraphics[width=3.4in]{graphs/step-1-statement.png}
	\caption{Statement-level Feature Extraction}
	\label{statement-level-feature-extraction}
\end{figure}

\subsection{Statement-level Feature Extraction}

For the statement level,  \tool uses the statement as the basic unit. And for each statement, \tool extracts three key features, including the code coverage information, sub-AST, and variables to represent the statement. 1> As for the code coverage information, \tool runs the relevant test cases for the input project. For the test case $t_i$, if it passes the statement $s_i$, \tool uses $c_i = 1$ to represent it while if it does not pass the statement $s_i$, \tool uses $c_i = 0$ to represent it. By linking all $c_i$ together, \tool can get $C = <c_1, c_2, ..., c_i>$. Also, \tool uses $r_i = 1$ to represent the the condition $passed$ and uses $r_i = 0$ to represent the condition $failed$ for the test case $t_i$ . By linking all $r_i$ together, \tool can get $R = <r_1, r_2, ..., r_i>$. By concatenate $C$ and $R$, \tool extract the code coverage information feature by $V_{coverage} = <c_1, c_2, ..., c_i, r_1, r_2, ..., r_i>$. The code coverage information feature in Figure \ref{statement-level-feature-extraction} shows an example of it. 2> For the sub-AST, it is very similar to the method level. By using JDT package, \tool can extract the AST for the whole method. And then, \tool searches for the nodes that appears in the statement. By collecting all these nodes and the edges between then, \tool can extract the sub-AST for the statement. 3> For the variables feature, for each statement, \tool collects all the variables $V$ that appeared in it and for each variable $v$ in $V$, \tool uses the $(variable_name variable_type)$ to represent it. Then \tool links all variables $V$ together with $,$ as a sequence $Seq_s$ as one of the static feature that \tool collect from the source code to represent the statement. For example, in Figure \ref{statement-level-feature-extraction}, \tool goes through the statement at $line 5$ in the input method and finds that only one variable $tree$ appears in the statement at $line 5$. So the variable feature here should be like $variable_name variable_type$ where $variable_name$ is $tree$ and $variable_type$ is $BSPTree Euclidean2D$.

With these three features, similar to the method-level, \tool builds three types of edges to link the statements together as a graph. The first type of edge is the program dependency edge (PD). \tool builds the PD by using the tool soot \cite{soot} for the method $m$. For example, in Figure \ref{statement-level-feature-extraction}, the blue edges are the PD edges and they show that in method $m$, statement at $line 4$ controls the statement at $line 5$. And statement at $line 5$ can control the statements at $line 7-8$ and the statements at $line 10-11$. The second type of edge that \tool extracts is the execution flow $E_s^e$. The execution flow is the order that the failed test case $t_i$ went through in the method $m$. For example, the green dotted edges are the execution flow. Within it, we can see that the test case executes $S7-S8$ but did not go through $S10-S11$. Even though in the PD, $S5$ is linked by $S10-S11$. It means that when running the test case $t_i$, it will not pass the $S10$ and $S11$ because of the if checking in $S5$. The last type of edge $E_s^c$ is the co-change relation in the statement level. \tool collects the co-change information for the commit about the statements that changed together before in the current method $m$ and one commit. In Figure \ref{statement-level-feature-extraction}, the commit history shows that $S4$ and $S5$ used to be changed together before. Hence, there is an orange edge that represents the co-change relation between $S4$ and $S5$.


%\begin{itemize}
%	\item Graph: 
%	\begin{itemize}
	%	\item Program Dependency Graph (PDG): \tool builds the PDG by using the tool soot \cite{soot} for the method $m$ that contains the statements that \tool want to analyze. \tool uses the generated PDG as the base graph. Within this graph, there are two types of edges including data dependency and control dependency. Both of these two types of edges are the static edges.
	%	\item Execution Path: \tool collects the execution path of the failed test case $t_i$ within the method $m$ and adds them into the PDG by adding a new type of edge $E_s^e$. The new edge direction is the same as the execution order. This type of edge is the dynamic edge.
%		\item Co-change Information: Similar to the method-level, \tool collects the co-change information for the commit about the statements that changed together before in one commit and the current method $m$. As for adding the co-change information into the PDG, \tool also creates the two-directional edge $E_s^c$ similar to the method level. This type of edge is the static edge.
%	\end{itemize}
%	\item Node Features: 
%	\begin{itemize}
	%	\item Code Coverage Information: \tool runs the relevant test cases for the input java project. For the test case $t_i$, if it passes the statement $s_i$, \tool uses $c_i = 1$ to represent it while if it does not pass the statement $s_i$, \tool uses $c_i = 0$ to represent it. By linking all $c_i$ together as $C = <c_1, c_2, ..., c_i>$, $C$ is considered by \tool as one of the dynamic feature.
	%	\item Statement Structure: Similar to the method-level, \tool generates a sub abstract syntax tree $Tree_s$ for each statement $S$ by using JDT \cite{} package. Each tree $Tree_s$ represent the structure of the relevant statement $s$. This feature is one of the static features that \tool collects from the source code to represent the statement.
	%	\item Variables: For each statement, \tool collects all the variables $V$ that appeared in it and for each variable $v$ in $V$, \tool uses the $(variable_name variable_type)$ to represent it. Then \tool links all variables $V$ together with $,$ as a sequence $Seq_s$ as one of the static feature that \tool collect from the source code to represent the statement. For example, the variables in the statement in line 3 in Figure \ref{fig:motiv} include $root$ and $sourceMap$. \tool generates the feature for them as $root Node, sourceMap SourceMap$ where $root$ and $sourceMap$ are the names and $Node$ and $SourceMap$ are the types.
%	\end{itemize}	
%\end{itemize} 

\input{sections/approach_step2}
\section{Dual-Task Learning for FL}
\label{sec:dual-learning}

\begin{figure}[t]
	\centering
	\includegraphics[width=3.4in]{graphs/dual-learning-2.png}
        \vspace{-18pt}
	\caption{Dual-Task Learning Fault Localization}
	\label{dual-learning}
\end{figure}

% Tien
%To ensure the matching of a method and its corresponding statements,
%we build the pairs ($G_M$, $g_M$) of the method-level graph $G_M$ for
%each method $M$ and the statement-level graph $g_M$ for all the
%statements in $M$. To ensure the co-fixing connections among the buggy
%methods for the same bug, in $G_M$, we model the co-fixed methods of
%$M$ via {\em co-fixed relations}. At the output layer, we label the
%nodes corresponding the faulty methods and faulty statements that were
%{\em fixed together} in the training data. A method with any faulty
%statement is labeled as faulty.

This section explains our dual-task learning scheme in step 3
(illustrated in Figures~\ref{dual-learning}
and~\ref{cross-stitch}). In training, for each bug $B$ in the training
dataset, to ensure the matching of a method and its corresponding
statements, we build the pairs ($G_M$, $g_M$) of the method-level
graph $G_M$ (Figure~\ref{method-level-feature-learning}) for each
faulty method $M$ and the statement-level graph $g_M$
(Figure~\ref{statement-level-feature-learning}) for all the statements
in $M$. Note that, to ensure the co-fixing connections among the buggy
methods for the same bug $B$, we model the {\em co-fixed methods} of
$M$ via co-fixed relations in $G_M$
(Figure~\ref{method-level-feature-learning}). At the output layer, we
label those methods as {\em buggy/co-fixed}. The {\em co-fixed
  statements} within $g_M$ for the bug $B$ are also labeled as {\em
  buggy/co-fixed}. The pairs ($G_M$, $g_M$) are used as the input of
this dual-task learning model (Figure~\ref{dual-learning}). We process
all the faulty methods $M$ for each bug $B$. In prediction, for each
method $M^{*}$ in the project, we build the pair ($G_{M^{*}}$,
$g_{M^{*}}$) and feed it to the trained dual-task model. In the output
graphs, each node (representing a method or a statement) will be
labeled as faulty or not. The nodes with the {\em buggy/co-fixed}
labels in $g_{M^{*}}$ are the~co-fixed statements for the given
bug. Let us detail our dual-task learning.

%The input of this step includes two feature graphs $G_M$ and $G_S$
%for the method and statement levels.  each node in a feature graph is
%a vector computed for a method ($V_M$) or a statement ($V_S$)
%(Section~\ref{feature-learning:sec}). The output includes the output
%graphs for methods and statements.  In training, each node in an
%output graph has a faulty or non-faulty label.

%For the prediction, each node in an output graph will be predicted as
%faulty or non-faulty.



%After having the vectorized graph $G_m$ and $G_s$ for both the method-level and the statement-level features,

%in this step, \tool applies a dual learning fault localization model to extract the fault locations. So there are two main tasks in this step, including the method-level fault localization and the statement-level fault localization. So the input for this step is the two graphs $G_m$ and $G_s$, and the expected output is the prediction label for each node in these two graphs from these two tasks.

%Tien
\noindent {\bf Graph Convolution Network (GCN) for FL.} First, {\tool}
has two GCN models~\cite{kipf2016semi}, each for FL at the method and
statement levels. GCN processes the attributes of the nodes (vectors)
and their edges (relations) in feature graphs. Each GCN model has
$n-1$ pairs of a graph convolution layer (\code{Conv}) and a rectified
linear unit (\code{ReLU}). They are aimed to consume and learn the
characteristic features in the input feature graphs. The last pair of
each GCN model is a pair of a graph convolution layer (\code{Conv})
and a softmax layer (\code{SoftMax}). The \code{SoftMax} layer plays
the role of the classifier on whether a node for a method or a
statement is labeled as {\em buggy/co-fixed} or {\em non-buggy}.



\begin{figure}[t]
	\centering
	\includegraphics[width=1.8in]{graphs/cross-stitch-2.png}
        \vspace{-6pt}
	\caption{Dual-Task Learning via Cross-stitch Unit}
	\label{cross-stitch}
\end{figure}

\noindent {\bf Dual-Task Learning with Cross-stitch Units.} In a
regular GCN model, those above pairs of \code{Conv} and \code{ReLU}
are connected to one another. However, to achieve dual-task learning
between method-level and statement-level FL (\code{methFL} and
\code{stmtFL}), we apply a cross-stitch unit~\cite{misra2016cross} to
connect the two GCN models. The sharing of representations between
\code{methFL} and \code{stmtFL} is modeled by learning a linear
combination of the input features in both feature graphs $G_M$ and
$g_M$. At each of the \code{ReLU} layer of each GCN model
(Figure~\ref{cross-stitch}), we aim to learn such a linear combination
of the output from the graph convolution layers (\code{Conv}) of the
\code{methFL} and \code{stmtFL} models.

The top sub-network in Figure~\ref{dual-learning} gets direct
supervision from \code{methFL} and indirect supervision (through
cross-stitch units) from \code{stmtFL}. Cross-stitch units regularize
%both tasks
\code{methFL} and \code{stmtFL} by learning and enforcing shared
representations by combining feature maps~\cite{misra2016cross}.


%Tien
%The α values of a cross-stitch unit model linear combinations of
%feature maps. Their initialization in the range [0, 1] is important
%for stable learning, as it ensures that values in the output
%activation map (after cross-stitch unit) are of the same order of
%magnitude as the input values before linear combination.
%-----

\noindent {\bf Formulation.} Let us explain the mathematic foundation
of this scheme. For each pair of the GCN model, the outputs of the
\code{ReLU} layer, called the hidden states, are computed as follows:
\begin{equation}\label{eq:1}
	\hat{A} = D'^{-\frac{1}{2}}A'D'{-\frac{1}{2}}
\end{equation}

\begin{equation}\label{eq:2}
	H^{i} = \Delta(\hat{A}X^{i}W^{i})
\end{equation}
Where $A'$ is the adjacency matrix of each feature graph; $D'$ is the
degree matrix; $W^{i}$ is the weight matrix for layer $i$; $X^{i}$ is
the input for layer $i$; $H^{i}$ is the hidden state of layer $i$ and
the output from the \code{ReLU} layer; and $\Delta$ is the activation
function \code{ReLU}.
%$H_i$ is the output from the \code{ReLU} layer.
In a regular GCN, $H^{i}$ is the input of the next layer of GCN (i.e.,
the input of \code{Conv}).

In Figures~\ref{dual-learning} and~\ref{cross-stitch}, a cross-stitch
unit is inserted between the \code{ReLU} layer of the previous pair
and the \code{Conv} layer of the next one. The input of the
cross-stitch unit includes the outputs of the two \code{ReLU} layers:
$H_M^i$ and $H_S^i$ (i.e., the hidden states of those layers in
\code{methFL} and \code{stmtFL}). We aim to learn the linear
combination of both inputs of the cross-stitch unit, which is
parameterized using the weights~$\alpha$.
%
Thus, the output of the cross-stitch unit is computed as:
\begin{equation}\label{cross-stitch-formula}
	\begin{bmatrix}
		X_M^{i+1}\\
		X_S^{i+1}
	\end{bmatrix}
        =
        \begin{bmatrix}
		\alpha_{MM} &  \alpha_{MS} \\
		\alpha_{SM} &  \alpha_{SS}
	\end{bmatrix}
	\begin{bmatrix}
		H_M^{i}\\
		H_S^{i}
	\end{bmatrix}
\end{equation}
$\alpha$ is the trainable weight matrix; $X_M^{i+1}$ and
$X_S^{i+1}$ are the inputs for the $(i+1)^{th}$ layers of the GCNs at the
method and statement levels.

$X_M^{i+1}$ and $X_S^{i+1}$ contain the information learned from both
\code{MethFL} and \code{StmtFL}, which helps achieve the main goal for
dual learning to enhance the performance of fault localization on both
levels.

In general, $\alpha$s can be set. If $\alpha_{MS}$ and $\alpha_{SM}$
are set to zeros, the layers are made to be task-specific.  The
$\alpha$ values model linear combinations of feature maps. Their
initialization in the range [0,1] is important for stable learning, as
it ensures that values in the output activation map (after
cross-stitch unit) are of the same order of magnitude as the input
values before linear combination~\cite{misra2016cross}.




%units combine the activations from multiple networks and can be
%trained end-to-end.

%Specifically, \tool firstly builds two separate GCN models \cite{kipf2016semi} for the method-level fault localization and the statement-level fault localization. For the GCN model applied on the method-level, there are $i$ graph convolutional layer $Conv_1, Conv_2, ..., Conv_i$ as shown in Figure \ref{dual-learning} and after each graph convolutional layer $Conv_i$, there is $ReLU$ layer follows it. Considering a graph convolutional layer $Conv_i$ and the following $ReLU$ layer together as one big layer, the GCN model contains $i$ layers in total. The only special case is in the last layer. There is a $SoftMax$ layer following the graph convolutional layer instead of a $RuLU$ layer. Similar to the GCN model applied on the method-level, for the statement-level fault location, there is the other GCN model with $i$ layers. Each layer contains one graph convolutional layer $Conv'_i$ and one $Relu$ layer. And the $SoftMax$ layer replaces the $ReLU$ layer in the last layer of the GCN model.


%To achieve the information sharing between two layers as a dual learning framework, we use the cross-stitch unit \cite{misra2016cross} for help. To be more detailed, for each layer of GCN, it calculates the hidden status using the following formula.





%Where $A'$ is the adjacency matrix; $D'$ is the degree matrix; $W_i$ is the weight matrix for layer $i$; $X_i$ is the input for layer $i$; $H_i$ is the hidden status of layer $i$; and $\Delta$ is the activation function $ReLU$. $H_i$ here is the output from the $ReLU$ layer and will be regarded as the input of the next layer of GCN. The cross-stitch unit is suitable to be added here.

%As seen in Figure \ref{dual-learning}, after the $ReLU$ layers we have $H_i^m$ and $H_i^s$ as the method-level and the statement-level hidden status. By putting them all into the cross-stitch unit, we have:

%\begin{equation}\label{eq:3}
%	\begin{bmatrix}
%		W_{m,m} &  W_{m,s} \\
%		W_{s,m} &  W_{s,s}
%	\end{bmatrix}
%	\begin{bmatrix}
%		H_m^{i}\\
%		H_s^{i}
%	\end{bmatrix}=
%	\begin{bmatrix}
%		X_m^{i+1}\\
%		X_s^{i+1}
%	\end{bmatrix}
%\end{equation}

%Where $W$ is the trainable or preset weight matrix, in \tool, we make it as the trainable weights; $X$ is the input for the $i+1$ layer of GCN. So with the cross-stitch unit, \tool gets $X_m^{i+1}$ and $X_s^{i+1}$ in this step as the input for the $i+1$ layer instead of directly feed $H_i^m$ and $H_i^s$ into the $i+1$ layer as input. The $X_m^{i+1}$ and $X_s^{i+1}$ contains the information learned from both the method-level and the statement-level that can help achieve the main goal for dual learning to enhance the performance of fault localization on both levels.

%But one special situation \tool may face in the cross-stitch unit is that the size of the outputs $H_i^m$ and $H_i^s$ from layer $i$ may be different. The different size of matrix will make the cross-stitch unit not work as expected.

If the sizes of the $H_M^{i}$ and $H_S^{i}$ are different, we need to adjust the sizes of the matrices. From Formula~\ref{cross-stitch-formula}, we have:
\begin{equation}\label{eq:4}
	X_M^{i+1} = \alpha_{MM}H_M^{i} + \alpha_{MS}H_S^{i}
\end{equation}
\begin{equation}\label{eq:5}
  X_S^{i+1} = \alpha_{SM}H_M^{i} + \alpha_{SS}H_S^{i}
\end{equation}

%Within formula \ref{eq:4} and \ref{eq:5}, \tool would like to resize $H_s^{i}$ in formula \ref{eq:4} and resize $H_m^{i}$ in formula \ref{eq:5}.

We resize $H_s^{i}$ in Formula~\ref{eq:4} and resize $H_m^{i}$ in
Formula~\ref{eq:5} if needed. We use the {\em bilinear interpolation}
technique~\cite{bilinear-interpolation} in image processing for
resizing. We pad zeros to the matrix to make the aspect ratio 1:1. If
the size needs to be reduced, we do the center crop on the matrix to
match the required size.

{\tool} also has a trainable threshold for \code{SoftMax} to classify
if a node corresponding to a method or a statement is faulty or not.

%By solving this problem, the cross-stitch unit can help share the information between two GCN models for both the method-level and the statement-level fault localization. The dual-learning fault localization model accepts the vectorized graphs as input and generates the label for each node. And there is a trainable threshold to determine if the node belongs to the $buggy$ class or the $non-buggy$ class. Thus, by collecting all nodes marked as $buggy$, \tool regards this set of predicted $buggy$ methods/statements as the output.


%\section*{Acknowledgments}
%This work was supported in part by the US National Science Foundation
%(NSF) grants CCF-1723215, CCF-1723432, TWC-1723198, CCF-1518897, and
%CNS-1513263.

\newpage

\balance

%\bibliographystyle{plain}
%\bibliographystyle{ACM-Reference-Format}
\bibliographystyle{ACM-Reference-Format}

\bibliography{References}

\end{document}
