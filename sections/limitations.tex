%\subsubsection{Limitations}
%\label{sec:limitations}

\vspace{3pt}
\noindent {\bf Limitations.}
%{\tool} has the following limitations.
First, our tool does not detect well the sets with +5 CC fixing statements
since it does not learn well those large co-changes. Second, it
does not work well in locating a fault that require only adding
statements to fix (all baselines do not work either).
%Third, the quality of test cases has large impact on
%{\tool}.
Third, if the faulty statements/methods occur far from the crash
method in the execution traces, it is not effective.
Finally, it does not have any mechanism to integrate program analysis
in expanding the faulty statements having dependencies with the
detected faulty ones.

%stack trace and execution trace

%The quality of test cases is important for our approach. If there are
%only a couple of passing test cases or the crash occurs far apart from
%the faulty method, {\tool} does not learn a useful representation
%matrix to localize the faults. 

%Moreover, it does not work well for short methods, as they provide
%less statement dependencies. It is also hard for our model to localize
%the uncommon faults. Because it is DL-based, if there is a very
%uncommon fault that may not be seen in the training dataset, it will
%not work correctly.
