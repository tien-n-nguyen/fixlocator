\subsection{\bf RQ1. Comparison Results with State-of-the-Art DL-based FL Approaches}
\label{sec:rq1-result}



Table~\ref{fig:rq1-details} shows {\em how well {\tool}'s coverage is
on the actual correct CC fixing statements (recall)}.
%{\em how well the coverage (recall)} of
%the results is on the actual CC fixing statements.
The result is w.r.t. the bugs in the oracle with {\em different
numbers $K$ of CC fixing statements}: $K$= $\#$$CC$-$Stmts$ = 1, 2, 3, 4,
5, and 5+. For example, in the oracle, there are 90 bugs with 3 faulty
statements. {\tool}'s predicted set correctly contains all 3 buggy
statements for 21 bugs (Hit-All), 2 of them for 25 bugs, and 1 faulty
statement for 51 bugs. As seen, regardless of $N$, {\tool} performs
better in any Hit-$N$ over the baselines for all $K$s. Note that
Hit-All = Hit-$N$ when $N$(\#overlaps) = $K$(\#CC-Stmts).


%As seen, {\tool} improves over the baselines in any metric when
%dealing with any number of faulty statements. For the bugs with
%$\leq$5 faulty statements, {\tool} locates more faulty statements than
%any baseline.


\begin{table}[t]
	\caption{RQ1. Detailed Comparison w.r.t. Faults with Different \# of CC Fixing Statements in an Oracle Set (Recall)}
	\vspace{-10pt}
        \tabcolsep 2pt
	{\small
		\begin{center}
			\renewcommand{\arraystretch}{1}
			\begin{tabular}{p{1.5cm}<{\centering}|p{0.8cm}<{\centering}|p{1cm}<{\centering}|p{0.8cm}<{\centering}|p{1.3cm}<{\centering}|p{0.9cm}<{\centering}|p{1.1cm}<{\centering}}
				\hline
				\#CC-Stmts in Oracle & Metrics & CNN-FL & DeepFL & DeepRL4FL & DEAR & {\textsc{Fix\-Locator}\xspace} \\
				\hline
				\multirow{1}{*}{1 (199 bugs)}   & Hit-1  & 78 & 76 & 84 & 74& 93 \\
%				& EXAM          & 0.16 & 0.16 & 0.15 & 0.08 \\
				\hline
				\multirow{2}{*}{2 (142 bugs)}  & Hit-1   & 67 & 64 & 70 &65 & 75 \\
				& Hit-2         & 33 & 30 & 34 &28 & 41 \\
%				& EXAM          & 0.15 & 0.18 & 0.17 & 0.11 \\
				\hline
				\multirow{3}{*}{3 (90 bugs)}  & Hit-1    & 46 & 44 & 47 &42 & 51 \\
				& Hit-2     & 21 & 20 & 23 & 20& 25\\
				& Hit-3     & 11 &10 & 13 &12 & 21 \\
%				& EXAM          & 0.17 & 0.19 & 0.19 & 0.12 \\
				\hline
				\multirow{4}{*}{4 (78 bugs)}  & Hit-1    & 41 & 42 & 42 &40 & 45 \\
				& Hit-2     &22 & 19 & 21 &20 & 24 \\
				& Hit-3     & 9 & 7 & 8 &5 & 12 \\
				& Hit-4     & 3 & 2 & 4 &2 & 9 \\
%				& EXAM          & 0.1 & 0.13 & 0.11 & 0.09 \\
				\hline
				\multirow{5}{*}{5 (43 bugs)}  & Hit-1    & 15 & 14 & 16 &13 & 18 \\
				& Hit-2     & 9 & 8 & 9 &7 & 12 \\
				& Hit-3     & 6 & 5 & 6 &5 & 7 \\
				& Hit-4     & 3 & 2 & 3 & 2& 3 \\
				& Hit-5     & 1 & 1 & 1 &0 & 1 \\
%				& EXAM          & 0.12 & 0.14 & 0.14 & & 0.10 \\
				\hline
				\multirow{6}{*}{5+ (283 bugs)}  & Hit-1 & 85 & 91 & 93 & 87 & 105 \\
				& Hit-2     & 40 & 42 & 45 & 41& 65 \\
				& Hit-3     & 31 & 34 & 37 & 32& 42 \\
				& Hit-4     & 17 & 14 & 21 & 15& 34 \\
				& Hit-5     & 4 & 3 & 5 & 2& 8 \\
				& Hit-5+    & 1 & 2 & 3 & 1& 3 \\
%				& EXAM          & 0.14 & 0.20 & 0.13 & 0.11 \\
				\hline
			\end{tabular}
			
			\label{fig:rq1-details}
		\end{center}
	}
\end{table}

\begin{table}[t]
	\caption{RQ1. Comparison Results with DL-based FL Models}
	\vspace{-10pt}
	{\small
		\begin{center}
			\renewcommand{\arraystretch}{1}
			\begin{tabular}{p{1.2cm}<{\centering}|p{1cm}<{\centering}|p{0.8cm}<{\centering}|p{1.2cm}<{\centering}|p{0.8cm}<{\centering}|p{1.3cm}<{\centering}}
				\hline
				Metrics & CNN-FL & DeepFL & DeepRL4FL & DEAR & {\tool} \\			
				\hline
				Hit-1   & 332 & 331 & 352 & 321 & 387 \\
				Hit-2	& 125 & 119 & 132 & 106 & 167 \\
				Hit-3	& 57 & 56 & 64 & 54 & 82 \\
				Hit-4	& 23 & 18 & 28 & 19 & 46 \\
				Hit-5	& 5 & 4 & 6 & 2 & 9 \\
				Hit-5+	& 1 & 2 & 3 & 1 & 3 \\
                                Hit-All & 127  & 121  & 139  & 115 & 168\\
%				EXAM    & 0.14 & 0.18 & 0.15 & 0.09 \\
				\hline
			\end{tabular}
			
			\label{fig:rq1-overview}
		\end{center}
	}
\end{table}

Table~\ref{fig:rq1-overview} shows the summary of the comparison
results in which we sum all the corresponding Hit-$N$ values across
different numbers $K$ of CC fixing statements in
Table~\ref{fig:rq1-details}.
%
%As seen in Table~\ref{fig:rq1-0}, the comparison results show that
%{\tool} can improve over all baselines on locating CC fixing
%statements in every evaluation metric. Particularly,
As seen, {\tool} can improve CNN-FL, DeepFL, DeepRL4FL, and DEAR by
16.6\%, 16.9\%, 9.9\%, and 20.6\%, respectively, in terms of Hit-1
(i.e., the predicted set contains at least one faulty statement).
%
%Hit-N@Set with N$\geq$2 aims to evaluate the capability of detecting
%multiple CC fixing statements. As seen, {\tool} locates more multiple
%statements in a set than any baseline. Specifically,
%
It also improves over those baselines by 33.6\%, 40.3\%, 26.5\%, and
57.5\% in terms of Hit-2, 43.9\%, 46.4\%, 28.1\%, and 51.9\% in terms
of Hit-3, 100\%, 155.6\%, 64.5\%, and 142.1\% in terms of Hit-4. Note:
Any Hit-$N$ reflects the cases of multiple CC statements. For example,
Hit-1 might include the bugs with more than one buggy/fixed statement.
%
Importantly, our tool produced the exact-match sets for 168/864 bugs
(19.5\%), relatively improving over the baselines 32\%, 38.8\%,
20.8\%, and 46.1\% in Hit-All. It performs well in Hit-All
when the number of CC statements $K$=1-4. However, producing the
exact-matched sets for all statements when $K \geq 5$ is still
challenging for all the models.
%Tien removed this para
%The sizes of our predicted sets range from 1
%to 8 statements with a median size of 3 and a mean size of 3.48. 75\%
%of the sets have a size from 3-5 statements. Thus, {\tool} detects CC
%fixing locations well for the common cases in terms of the number of
%faulty statements, while it is still challenging for all models for
%the cases of more than 5 faulty statements.

%Thus, Hit-5@Set and Hit5+@Set are low as they are already larger than
%our predicted sets. However, all other baselines also cannot work on
%the cases with N$\geq$5.

%Tien removed this EXAM para
%{\tool} also reduces EXAM scores compared to the baselines CNN-FL,
%DeepFL, and DeepRL4FL by 22.2\%, 30\%, and 22.2\%, respectively. Thus,
%{\tool} saves more human efforts in searching for the CC fixing
%statements than the baselines.

%----------- PRECISION ---------

\begin{table}[t]
	\caption{RQ1. Detailed Comparison w.r.t. Faults with Different \# of CC Fixing Statements in a Predicted Set (Precision)}
\vspace{-10pt}
        \tabcolsep 2pt
	{\small
		\begin{center}
			\renewcommand{\arraystretch}{1}
			\begin{tabular}{p{1.5cm}<{\centering}|p{0.8cm}<{\centering}|p{1cm}<{\centering}|p{0.8cm}<{\centering}|p{1.2cm}<{\centering}|p{0.9cm}<{\centering}|p{1.1cm}<{\centering}}
				\hline
				\#Stmts in Predicted Set & Metrics & CNN-FL & DeepFL & DeepRL4FL & DEAR & {\textsc{Fix\-Locator}\xspace} \\
				\hline
				\multirow{1}{*}{1 (203 bugs)}   & Hit-1  & 83 & 79 & 87 & 75 (183)& 99 \\
%				& EXAM         						& 0.17 & 0.19 & 0.17 & 0.08 \\
				\hline
				\multirow{2}{*}{2 (165 bugs)}  & Hit-1   & 75 & 72 & 78 & 71 (172) & 83 \\
				& Hit-2       						& 36 & 34 & 39 & 34 (172) & 45 \\
%				& EXAM          					& 0.15 & 0.16 & 0.14 & 0.12 \\
				\hline
				\multirow{3}{*}{3 (120 bugs)}  & Hit-1    & 52 & 46 & 48 & 41 (129) & 55 \\
				& Hit-2         					& 24 & 22 & 26 & 19 (129) & 27\\
				& Hit-3         				  	& 12 &11 & 14 & 10 (129) & 23 \\
%				& EXAM         						 & 0.16 & 0.20& 0.17 & 0.13 \\
				\hline
				\multirow{4}{*}{4 (96 bugs)}  & Hit-1    & 47 & 49 & 46 & 33 (78) & 51 \\
				& Hit-2        				    	 &24 & 21 & 22 & 14 (78) & 26 \\
				& Hit-3       					    & 11 & 9 & 10 & 5 (78) & 14 \\
				& Hit-4       					    & 5 & 3 & 6 & 1 (78) & 11 \\
%				& EXAM        					    & 0.11 & 0.12 & 0.12 & 0.10 \\
				\hline 
				\multirow{5}{*}{5 (73 bugs)}  & Hit-1    & 17 & 16 & 17 & 12 (55) & 19 \\
				& Hit-2       						  & 10 & 10 & 11 & 7 (55) &14 \\
				& Hit-3       						  & 8 & 6 & 7 & 4 (55) &9 \\
				& Hit-4         					  & 3 & 3 & 4 & 1 (55) &5 \\
				& Hit-5      						   & 2 & 1 & 2 & 0 (55) &2 \\
%				& EXAM       						   & 0.12 & 0.15 & 0.16 & & 0.09 \\
				\hline
				\multirow{6}{*}{5+ (178 bugs)}  & Hit-1 & 58 & 69 & 76 & 68(218)& 80 \\
				& Hit-2        				       & 31 & 32 & 34 & 32 (218) &55 \\
				& Hit-3       				       & 26 & 30 & 33 & 24 (218) &36 \\
				& Hit-4      				       & 15 & 12 & 18 & 16 (218)&30 \\
				& Hit-5      				       & 3  & 3 & 4 & 5 (218) &7 \\
				& Hit-5+       				       & 1  & 2 & 3 & 2 (218) & 3 \\
%				& EXAM        				       & 0.13 & 0.20 & 0.12 & 0.10 \\
				\hline
			\end{tabular}
			\label{fig:rq1-prec}
		\end{center}
	}
\end{table}

Table~\ref{fig:rq1-prec} shows the comparison on {\em how precise the
results are} in a predicted set. For example, when {\em the number of the
CC statements in a predicted set} is $K'$=3, there are 23 bugs in which
all of those 3 faulty statements are correct (there might be other
statements missing).  There are 27 bugs in which two of the 3
predicted, faulty statements are correct. There are 55 bugs in which
only one of the 3 predicted, faulty statements are correct. As seen,
regardless of $N$, {\tool} is more precise than the baselines for all
$K'$s.



%tips:

%1. The exam and exme\_avg score is the lower the better

%2. The hit@set is the higher the better

%2. \tool is the best performed approach



\begin{table}[t]
  \caption{RQ1. Comparison with Baselines w.r.t. Ranking}
  \tabcolsep 4.8pt
        \vspace{-9pt}
	{\small
		\begin{center}
			\renewcommand{\arraystretch}{1}
			\begin{tabular}{p{1.3cm}<{\centering}|p{0.3cm}<{\centering}|p{0.3cm}<{\centering}|p{0.3cm}<{\centering}|p{0.2cm}<{\centering}|p{0.2cm}<{\centering}||p{0.3cm}<{\centering}|p{0.3cm}<{\centering}|p{0.3cm}<{\centering}|p{0.2cm}<{\centering}|p{0.2cm}<{\centering}|p{0.2cm}<{\centering}}
				\hline
				    & \multicolumn{5}{c||}{Hit-N@Top-5}& \multicolumn{6}{c}{Hit-N@Top-10}\\
				\cline{2-12} N= 
											 &1&2&3&4&5&1&2&3&4&5&5+\\
				
				\hline
				CNN-FL      & 533 & 311 & 133 & 33 & 4 & 578 & 386 & 166 & 42 & 10 & 81 \\
				DeepFL		& 525 & 298 & 131 & 35 & 6 & 563 & 364 & 156 & 42 & 10 & 83 \\
				DeepRL4FL	& 586 & 339 & 159 & 32 & 9 & 623 & 407 & 186 & 48 & 13 & 92 \\
                                DEAR	& 501 & 274 & 119 & 25 & 3 & 544 & 341 & 142 & 36 & 7 & 71 \\
				\hline
				\tool       & 633 & 420 & 195 & 46 & 11& 690 & 470 & 217 & 51 & 13 & 94 \\
				\hline
			\end{tabular}
			
			\label{fig:rq1-rank}
		\end{center}
	}
\end{table}

Table~\ref{fig:rq1-rank} shows the comparison as ranking is considered
(Hit-N@Top-$K$).  As seen,
%{\tool} can locate more faulty statements than any baseline in
%Hit-N@Top-5 and Hit-N@Top-10. The results indicate that
in the ranking setting, {\tool} locates more CC fixing
statements than any baseline.
%locates more faulty statements than any baselines, especially dealing
%multiple faulty statements.
For example, {\tool} improves the best baseline DeepRL4RL
by 23.9\% in Hit-2@Top-5, 22.6\% in Hit-3@Top-5, 43.8\% in
Hit-4@Top-5, and 22.2\% in Hit-5@Top-5, respectively. The same trend
is for Hit-N@Top-10.

%The Hit-N@Top-10 result has the same trend as Hit-N@Top-5 in which
%{\tool} improves over the baselines.

We did not compare with the spectrum-/mutation-based FL
models since DeepRL4FL~\cite{icse21-fl} was shown to outperform
them.
%Thus, our result shows that {\tool} could improve over those
%spectrum-/mutation-based FL approaches.

%\begin{table}[t]
%	\caption{RQ1. Detailed Comparison Results on Each Type of Faults Classified using \# of CC Fixing Statements. (T5: Top-5, T10: Top-10)}
%	\vspace{-6pt}
%	{\small
%		\begin{center}
%			\renewcommand{\arraystretch}{1}
%			\begin{tabular}{p{0.8cm}<{\centering}|p{0.8cm}<{\centering}|p{0.35cm}<{\centering}|p{0.35cm}<{\centering}|p{0.3cm}<{\centering}|p{0.3cm}<{\centering}|p{0.5cm}<{\centering}|p{0.5cm}<{\centering}|p{0.4cm}<{\centering}|p{0.4cm}<{\centering}}
%				\hline
%				\multirow{2}{*}{\#Stmts} & \multirow{2}{*}{Metrics} &\multicolumn{2}{c|}{CNN-FL} & \multicolumn{2}{c|}{DeepFL} & \multicolumn{2}{c|}{DeepRL4FL} & \multicolumn{2}{c}{\tool} \\
%				\cline{3-10}
%				& & T5&T10&T5&T10&T5&T10&T5&T10 \\
%				\hline
%				\multirow{3}{*}{1 (199)}   & Hit-1  & 120 &147&110&140&144&143&157&177 \\
%				& EXAM                		        & 0.16&0.16&0.16&0.16&0.15&0.15&0.08&0.08 \\
%				\hline
%				\multirow{4}{*}{2 (142)}  & Hit-1   & 118 &128&106&120&105&130&124&122\\
%				& Hit-2                 		    & 77 &100&69&101&84&104&94&111\\
%				& EXAM       		                &  0.15&0.15&0.18&0.18&0.17&0.17&0.11&0.11 \\
%				\hline
%				\multirow{5}{*}{3 (90)}  & Hit-1    & 72 &74&71&72&79&81&80&95 \\
%				& Hit-2                			    & 56 &67&55&55&57&65&68&75\\
%				& Hit-3               			    & 27 &32&22&28&33&40&54&56 \\
%				& EXAM                         	    &  0.17&0.17&0.19&0.19&0.19&0.19&0.12&0.12\\
%				\hline
%				\multirow{6}{*}{4 (78)}  & Hit-1    & 70 &72&62&66&67&75&74&81 \\
%				& Hit-2  						    & 50 &69&44&60&52&60&54&65 \\
%				& Hit-3   						    & 20 &29&17&18&19&25&26&32 \\
%				& Hit-4   						    & 4 &6&4&5&4&7&9&10\\
%				& EXAM    					        &  0.10&0.10&0.13&0.13&0.11&0.11&0.09&0.09 \\
%				\hline
%				\multirow{7}{*}{5 (43)}  & Hit-1    & 22 &28&22&24&26&27&31&31 \\
%				& Hit-2    						    & 23 &25&18&26&22&29&28&36 \\
%				& Hit-3  						    & 13 &17&11&15&14&16&18&19\\
%				& Hit-4    						    & 4 &5&4&5&3&5&3&6 \\
%				& Hit-5    						    & 1 &2&2&3&1&2&1&1 \\
%				& EXAM       					    &  0.12&0.12&0.14&0.14&0.14&0.14&0.10&0.10 \\
%				\hline
%				\multirow{8}{*}{5+ (283)}  & Hit-1  & 131 &129&154&141&165&167&167&184 \\
%				& Hit-2      					    & 105 &125&112&122&124&149&176&183 \\
%				& Hit-3   						    & 73 &88&81&95&93&105&97& 110\\
%				& Hit-4     						& 25 &31&27&32&25&36&34&35 \\
%				& Hit-5    						    & 3 &8&4&7&8&11&10&12\\
%				& Hit-5+    						& 0 &81&0&83&0&92&0&94 \\
%				& EXAM         					    &  0.14&0.14&0.20&0.20&0.13&0.13&0.11&0.11 \\
%				\hline
%			\end{tabular}
%			
%			\label{fig:rq1-1}
%		\end{center}
%	}
%\end{table}

% Table generated by Excel2LaTeX from sheet 'Sheet1'
% Table generated by Excel2LaTeX from sheet 'Sheet1'
\begin{table}[t]
  \centering
  \caption{Overlapping Analysis Results for Hit-1}
  \vspace{-9pt}
  \small
    %\begin{tabular}{|l|r|r|r|}
    \begin{tabular}{l|c|c|c}\hline
    	
    %\toprule
          & \multicolumn{3}{c}{{\tool}} \\ \cline{2-4}
    %\midrule
         % & \multicolumn{1}{l|}{Unique-Baseline} & \multicolumn{1}{l|}{Overlap} & \multicolumn{1}{l|}{Unique-{\tool}} \\
         
         & Unique-Baseline & Overlap& Unique-{\tool} \\\cline{1-4}
   % \midrule
    CNN-FL & 48    & 284   & 103 \\\hline
   % \midrule
    DeepFL & 54    & 277   & 110 \\\hline
    %\midrule
    DeepRL4FL & 61    & 291   & 96 \\\hline
    %\midrule
    DEAR  &   35    &   286    & 101 \\\hline
    %\bottomrule
    \end{tabular}%
  \label{tab:overlap}%
\end{table}%



We also performed the analysis on the overlapping between the results
of {\tool} and each baseline. As seen in Table~\ref{tab:overlap},
{\tool} can detect at least one correct faulty statement in 103 bugs
that CNN-FL missed, while CNN-FL can do so only in 48 bugs that
{\tool} missed.
%CNN-FL can detect at least one correct faulty statement in 48 bugs
%that {\tool} missed, while {\tool} can do so in 103 bugs that CNN-FL
%missed.
Both {\tool} and CNN-FL can do so in the same 284 bugs. In
brief, {\tool} can detect at least one correct buggy statement in more
``unique'' bugs than any baseline.


%For the Hit-1 results (predicted set has $\geq$1 correct buggy
%statement):

%1. CNN-FL(Unique): 48 bugs, {\tool}(Unique): 103, Overlapping: 284 

%2. DeepFL(Unique): 54, {\tool}(Unique): 110, Overlapping: 277

%3. DeepRL4FL(Unique): 61, {\tool}(Unique): 96, Overlapping: 291

%4. DEAR (Unique): {\bf xx}, {\tool}(Unique): {\bf xx}, Overlapping: {\bf xxx}

%For the Hit-1@Top-5 results (predicted list has 5 buggy statements):

%1. CNN-FL(Unique): 37, {\tool}(Unique): 137, Overlapping: 496

%2. DeepFL(Unique): 43, {\tool}(Unique): 151, overlapping: 482

%3. DeepRL4FL(Unique): 62, {\tool}(Unique): 109, overlapping: 524

%4. DEAR (Unique): {\bf xx}, {\tool}(Unique): {\bf xx}, Overlapping: {\bf xxx}
