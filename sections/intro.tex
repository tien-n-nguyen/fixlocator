\section{Introduction}

Detecting and fixing software defects is crucial in a software
development process.
To reduce the efforts from developers in that
process, several {\em fault localization} (FL)
approaches~\cite{fl-survey} have been introduced to help localize the
source of the fault that needs to be fixed. The input of an FL model
is the execution of a test suite, in which some of the test cases are
passing or failing ones. Specifically, the key input is the {\em code
  coverage matrix} in which the rows and columns correspond to the
statements and test cases, respectively.  Each cell is assigned with
the value of 1 if the statement is executed in the
respective test case, and with the value of 0, otherwise.  An FL model
uses such information to identify the list of {\em suspicious lines of
  code} that are ranked based on their associated {\em suspiciousness
  scores}~\cite{fl-survey}. In recent advanced FL, several approaches
also support fault localization at method
level~\cite{DeepFL,icse21-fl}.



%In the FL problem, given the execution of test cases, an FL tool
%identifies the set of {\em suspicious lines of code} with their
%associated suspiciousness scores~\cite{fl-survey}.  The key input of
%an FL tool is the {\em code coverage matrix} in which the rows and
%columns correspond to the source code statements and test cases,
%respectively.  Each cell is assigned with the value of 1 if the
%respective statement is executed in the respective test case, and with
%the value of 0, otherwise. In recent FL, several researchers also
%advocate for fault localization at method level~\cite{DeepFL}. FL at
%both levels are useful for developers.

The FL approaches can be broadly divided into the following
categories: {\em spectrum-based fault localization} (SBFL)
approaches~\cite{abreu2006evaluation,jones2001visualization,keller2017critical},
{\em mutation-based fault localization} (MBFL)
approaches~\cite{MUSE,papadakis2012using,Metallaxis}, and {\em machine
  learning (ML)} and {\em deep learning (DL)}~\cite{DeepFL,icse21-fl}.
For SBFL approaches, the key idea is that a line covered more in the
failing test cases than in the passing ones is more suspicious than a
line executed more in the passing ones.
%
To improve SBFL, MBFL
approaches~\cite{MUSE,papadakis2012using,Metallaxis} enhance the code
coverage information by modifying a statement with mutation operators,
and then collecting code coverage when executing the mutated programs
with the test cases. The MBFL approaches apply suspiciousness score
formulas in the same manner as in SBFL approaches on the matrix for
each original statement and its mutated code.
%
Finally, ML and DL-based FL approaches explore the code coverage
matrix and apply different neural network models for fault
localization.


%{\em Spectrum-based fault localization} (SBFL)
%approaches~\cite{abreu2006evaluation,jones2001visualization,keller2017critical}
%take the recorded lines of code that were covered by each of the given
%test cases, and assigned each line of code a suspiciousness score
%based on the code coverage matrix. Despite using different
%formulas to compute that score, the idea is that a line covered more
%in the failing test cases than in the passing ones is more suspicious
%than a line executed more in the passing ones. A key drawback of those
%approaches is that the same score is given to the lines that have been
%executed in both failing and passing test cases. An example is the
%statements that are part of a block statement and executed at the
%same nested level. Another example is the conditions of the
%condition statements, e.g., \code{if}, \code{while}, \code{do},
%and \code{switch}.

%To improve SBFL, {\em mutation-based fault localization} (MBFL)
%approaches~\cite{MUSE,papadakis2012using,Metallaxis}
%enhance the code coverage information by modifying a statement with
%mutation operators, and then collecting code coverages when executing
%the mutated programs with the test cases. They apply suspiciousness
%score formulas in the same manner as the spectrum-based FL approaches on
%the code coverage matrix for each original statement and its mutated
%ones. Despite the improvement, MBFL are not effective for the bugs
%that require the fixes that are more complex than a mutation
%(Section~\ref{motivexample}).

%{\em Machine learning (ML)} and {\em deep learning (DL)}
%have been used in fault localization. DeepFL~\cite{DeepFL}
%computes for each faulty method a vector with +200 scores in which
%each score is computed via a specific feature, e.g., a spectrum-based
%or mutation-based formula, or a code complexity metric. Despite its
%success, the accuracy of DeepFL is still limited. A reason could be
%that it uses various calculated scores from different formulas as a
%proxy to learn the suspiciousness of a faulty element, instead of
%fully exploiting the code coverage. Some formulas, such
%as the spectrum- and mutation-based formulas, inherently suffer from
%the issues as explained earlier with the statements covered by
%both failing and passing test cases.

Despite their successes, the state-of-the-art FL approaches are still
limited in locating all dependent fixing locations that need to be
repaired at the same time in the same fix. In practice, there are
several bugs that require {\em dependent changes in the same fix to
  multiple lines of code in one or multiple hunks of the same or
  different methods for the program to pass the test cases}.
%In real-world software development, there are several bugs that
%require a fix to multiple lines of code in one or multiple hunks of
%the same or different methods. {\em The fixing changes to those lines
%  of code are dependent to one another and need to be made in the same
%  fix for the program to pass the test cases}.
For those bugs, applying the fixing change to one statement at a time
will not make the program pass the test case after the change to one
statement. This capability to detect the fixing locations of the
co-changes in a fix for a bug (let us call them {\em Co-change (CC)
  Fixing Locations}) is important for both the manual process of bug
fixing as well as for the automated process of program repair
(APR). We focus on CC fixing statements in this work. For a manual
process, such capability {\em saves effort and time~for developers} in
locating all the faulty statements that need to be fixed at the same
time. For APR, such capability will enable an APR model to {\em
make the correct and complete changes to fix a bug}.

%This capability to detect the fixing locations of the co-changes in a
%fix for a bug (let us call them {\em Co-change (CC) Fixing Locations})
%is important for both the manual process of bug fixing as well as for
%the automated process of program repair. For a manual process, such
%capability will {\em save effort and time for developers} in locating all
%the faulty statements that need to be fixed at the same time. For
%automated program repair (APR), such capability will enable an APR
%model to {\em correctly and completely make the changes to fix the bug}.

From the ranked list of suspicious statements returned from an
existing FL model, a naive approach to detect CC fixing locations would
be to take the top $k$ statements in that list and to consider them as
to be fixed together. This solution might not be effective
because the mechanisms used in the state-of-the-art FL approaches have
never considered the co-change nature of those fixes. Our empirical
evaluation also confirmed that (Section~\ref{sec:rq1-result}).

Detecting all the CC fixing locations at multiple statements in
potentially multiple methods is challenging. A naive solution would be
detecting the potential methods that need to be fixed together and
then detecting potential statements that need to be changed together
in each of those methods. However, doing so will create a confounding
effect from the inaccuracy of the detection of co-fixed methods to the
detection of the co-fixed statements.

We propose {\tool}, a fault localization approach to derive the
co-change fixing locations in the same fix for a fault (i.e, multiple
faulty statements in possible multiple faulty methods).
%for faulty statements/methods that can locate multiple faulty
%statements in possible multiple faulty methods.
To avoid the confounding effect in that naive solution, we treat this
problem as {\em dual-task learning} with two models. First, the {\em
  method-level FL} (\code{MethFL}) model learns the methods that need
to be modified in the same fix. Second, the {\em statement-level FL}
(\code{StmtFL}) model learns the co-fixed statements in the same or
different methods. The intuition is that they are closely related,
which we refer to as {\em duality}. {\em Correct learning for a model
  can benefit the other and vice versa}. If two statements in two
methods are fixed together for a bug, those methods are also
co-fixed. If two methods are co-fixed, some of their statements are
also co-fixed. Exploring this~duality can provide useful constraints
to detect CC fixing locations. Thus, instead of
cascading \code{MethFL} and \code{StmtFL}, we train them
simultaneously with soft-sharing the parameters of the models to
exploit this duality. Specifically, we leverage the cross-stitch
units~\cite{misra2016cross} to connect \code{MethFL} and
\code{StmtFL}. In a cross-stitch unit, the sharing of representations
between \code{MethFL} and \code{StmtFL} is modeled by learning a
linear combination of the input features from two models. The
cross-stitch units enable the {\em propagation of the impact of
  \code{MethFL} on \code{StmtFL} and vice versa}.

%We apply a probabilistic correlation as a regulization term in the
%loss function in the join training. We also design a novel constraint
%about the attention mechanism in the two models.

In addition to the new solution in dual-task learning, we explore a
novel feature for this CC fixing location problem: co-change
statements, that have never been considered in FL. The rationale is
that those co-changed statements in the past might become the
statements that will be fixed together in the future. Finally,
since~the co-fixed statements are often interdependent, we use
Graph-based Convolution Network (GCN)~\cite{li2019gcn} to integrate
different types of program dependencies among statements, e.g., data
and control dependencies, execution traces, stack traces, etc. We
also encode test coverage and co-changed and co-fixed statements in
the graph. The GCN model learns and predicts the bugginess of
the~statements.

%The statements that need to be fixed together are often dependent on
%one another with program dependencies. Thus, we use Graph-based
%Convolution Network (GCN)~\cite{li2019gcn} to model different types of
%dependencies among the statements. Thus, we encode both the program
%dependence graph (PDG) and execution path (EP) in our graph modeling.
%For convenience, we also encode the co-change relations among the
%statements into our general graph representations with different types
%of edges representing different relations. The GCN model enables both
%nodes’ and edges’ attributes and learns to classify the nodes.

%{\bf FIXME.} We conducted several experiments to evaluate {\tool} on Defects4J benchmark~\cite{defects4j}. Our empirical results show that \tool locates 245 faults and 71 faults at the method level and the statement level, respectively, using only top-1 candidate (i.e., the first ranked element is faulty). It can improve the top-1 results of the state-of-the-art \textit{statement-level} FL baselines by 317.7\%, 273.7\%, 173.1\%, 195.8\%, and 491.7\% when comparing with Ochiai~\cite{Ochiai}, Dstar~\cite{DStar} Muse~\cite{MUSE}, Metallaxis~\cite{Metallaxis}, and RBF-Neural-Network-based FL (RBF)~\cite{RBF_Neural_Network}, respectively.  {\tool} also improves the top-1 results of the existing \textit{method-level} FL baselines, MULTRIC~\cite{MULTRIC}, FLUCCS~\cite{FLUCCS}, TraPT~\cite{TraPT}, and DeepFL~\cite{DeepFL}, by 206.3\%, 53.1\%, 57.1\%, and 15.0\%, respectively. Our results show that three sources of information in {\tool} positively contribute to its high accuracy. We also evaluated {\tool} on ManyBugs~\cite{LeGoues15tse}, a~ben\-chmark of C code with 9 projects. The results are~consistent with the ones on Java code. {\tool} localizes 27 faulty statements and 98 faulty methods using only top-1 results.

We conducted several experiments to evaluate {\tool} on
Defects4J~\cite{defects4j}. Our empirical results show that {\tool}
improves over the baselines, CNN-FL~\cite{zhang2019cnn},
DeepFL~\cite{DeepFL}, and DeepRL4FL~\cite{icse21-fl} by 16.6\%,
16.9\%, and 9.9\%, respectively, in terms of Hit-1 (i.e., the number
of bugs in which the predicted set overlaps with the oracle set {\em
  at least} one faulty statement),
% Moreover, it can locate more CC fixing statements than CNN-FL,
%DeepFL, and DeepRL4FL
and by 33.6\%, 40.3\%, and 26.5\% in terms of Hit-2 (i.e., the number
of overlapping statements $\geq$2), 43.9\%, 46.4\%, and 28.1\% in terms
of Hit-3, and 100\%, 155.6\%, and 64.5\% in terms of Hit-4,
respectively.
%
{\tool} also improves over CNN-FL, DeepFL, and DeepRL4FL by {\bf
  xx.x\%}, {\bf xx.x\%}, and {\bf xx.x\%} in terms of Hit-All (i.e.,
the predicted set exactly matches with the set in oracle for a
bug). Moreover, it can reduce the EXAM scores of CNN-FL, DeepFL, and
DeepRL4FL by {\bf 22.2\%}, {\bf 30\%}, and {\bf 22.2\%},
respectively. EXAM score is the percentage of executed statements
needed to be examined before finding the first faulty statement in the
predicted set. The lower EXAM score shows that {\tool} can save more
human efforts in detecting CC fixing statements. To show its
usefulness, we used {\tool} to replace the original CC fixing-location
detection module of DEAR~\cite{icse22}, an automated program repair
tool. The result shows that {\tool} helps DEAR relatively improve {\bf
  xx.x}\% in terms of the numbers of bug fixes in comparison with that
original module in DEAR.


Through our ablation analysis on the impact of different features and
modules of {\tool}, we showed that all designed features/modules have
contributed to its high performance. Specifically, the proposed
dual-task learning significantly improves the statement-level FL by up
to {\bf 100\%} in terms of EXAM score. The designed feature of
co-change relations among methods and statements has also contributed
to {\tool}'s high accuracy level.


%245 faults and 71 faults at the method level and the statement level, respectively, using only top-1 candidate (i.e., the first ranked element is faulty). It can improve the top-1 results of the state-of-the-art \textit{statement-level} FL baselines by 317.7\%, 273.7\%, 173.1\%, 195.8\%, and 491.7\% when comparing with Ochiai~\cite{Ochiai}, Dstar~\cite{DStar} Muse~\cite{MUSE}, Metallaxis~\cite{Metallaxis}, and RBF-Neural-Network-based FL (RBF)~\cite{RBF_Neural_Network}, respectively.


%{\tool} also improves the top-1 results of the existing \textit{method-level} FL baselines, MULTRIC~\cite{MULTRIC}, FLUCCS~\cite{FLUCCS}, TraPT~\cite{TraPT}, and DeepFL~\cite{DeepFL}, by 206.3\%, 53.1\%, 57.1\%, and 15.0\%, respectively. Our results show that three sources of information in {\tool} positively contribute to its high accuracy. We also evaluated {\tool} on ManyBugs~\cite{LeGoues15tse}, a~ben\-chmark of C code with 9 projects. The results are~consistent with the ones on Java code. {\tool} localizes 27 faulty statements and 98 faulty methods using only top-1 results.

The contributions of this paper are listed as follows:

{\bf 1. {\tool}: Advancing DL-based fault localization} to derive the
{\bf CC fixing locations} (multiple faulty statements) in the same fix
for a bug. We treat that problem as {\bf {\em dual-task learning to
    propagate the impact}} between method- and statement-level~FL.

%with the joint training of the method-level and statement-level
%co-fixing learning models.

{\bf 2. Novel graph-based representation learning with novel co-change
  features for FL.} Our graph-based representation learning with GCN
and the novel type of features in co-changed statements enable the
dual-task models to derive CC fixing locations.

{\bf 3. Extensive empirical evaluation.} We evaluated {\tool} against
the recent DL-based FL models to show its performance and
usefulness in APR. Our data and tool are available
at~\cite{FixLocator2022}.

