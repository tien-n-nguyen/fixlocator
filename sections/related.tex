\section{Related Work}
%we need to talk about the newly fse lingming's paper, code not available, and also their results on benchmark is even lower than deepFL.
%has been intensively studied in the literature, e.g., Ochiai~\cite{abreu2006evaluation} and Jaccard~\cite{abreu2007accuracy}.

Several types of techniques have been proposed to locate faulty
statements/methods. However, none of the existing FL approaches detect
{\em CC fixing locations} as in {\tool}. A closely related work is
DeepRL4FL~\cite{icse21-fl}, which has two components: CC fixing
location detection and auto-fixing. Our experiment
(Section~\ref{sec:eval}) shows that {\tool} performed better than
DeepRL4FL's CC fixing location detection component and helps DeepRL4FL
improve bug-fixing. Another work is
Hercules~\cite{hercules-icse19}, an APR tool that can detect multiple
buggy hunks of code. Hercules can detect only the buggy hunks with
similar statements (called replicated fixes), while {\tool} detects
general CC fixing locations. In comparison, {\tool} and Hercules
detect 26 and 15 multi-hunk bugs~\cite{hercules-icse19}, respectively
among 395 bugs in Defects4J-v1.2. However, Hercules is not available
for running on Defects4J-v2.0, which is used in this work.

The Spectrum-based Fault Localization
(SBFL)~\cite{zhang2011localizing, abreu2007accuracy,
  jones2005empirical, abreu2006evaluation, naish2011model,
  wong2007effective, liblit-pldi05, lucia2014extended} and
Mutation-based Fault Localization (MBFL)~\cite{Metallaxis, MUSE,
  zhang2013injecting, budd1981mutation, zhang2010test, musco2017large}
have been extensively explored. SBFL and MBFL are for statement-level
fault localization. However, the key limitation of SBFL and MBFL is
that they cannot differentiate the statements with the same scores or
cannot have effective mutators to catch a fault that is complex than a
mutator. Recently, learning-based FL techniques have been developed to
improve SBFL and MBFL.
%\textbf{Machine Learning or Statistical Fault Localization}
A group of learning-to-rank
FL~\cite{MULTRIC,TraPT,b2016learning,sohn2017fluccs} approaches have
been proposed to locate faulty methods.
%they all at method level fault localization
A recent work combining statistical FL with casual inference
techniques~\cite{kuccuk2021improving} has been developed for
statement-level FL. However, all of the above approaches are designed
to perform a binary classification on statements or methods to rank
the suspicious candidates. Their goal limits their ability of locating
multiple CC fixing statements.

With advances in machine learning, researchers have used deep learning
for FL, which is the most relevant line of work to our work. Early
neural networks~\cite{zheng2016fault, briand2007using, zhang2017deep,
  wong2009bp} have been applied to FL.  However, they mainly work on
the test coverage information, which has clear limitations (e.g., it
cannot distinguish elements accidentally executed by failed tests and
the actual faulty elements)~\cite{TraPT}.

Some recent deep learning based approaches, such as
GRACE~\cite{lou2021boosting}, DeepFL~\cite{DeepFL},
CNNFL~\cite{zhang2019cnn}, DeepRL4FL~\cite{icse21-fl} can achieve
better results than the above types. The recent work GRACE proposes a
new graph representation for a method and learns to rank the faulty
methods, which is different from our work. {\tool} is designed for
locating multiple CC fixing statements for a fault. DeepFL and
DeepRL4FL can outperform the learning-based and early neural networks
FL techniques, such as MULTRIC~\cite{MULTRIC}, TrapT~\cite{TraPT}, and
Fluccs~\cite{sohn2017fluccs}. Through our empirical studies, we showed
that {\tool} can outperform the compared baselines: CNNFL, DeepFL, and
DeepRL4FL, in detecting CC fixing statements.


%Lingming's paper GRACE has code now, but we won't have time to implement it and also it is designed for method ranking, our approach is proposed for statement-level. Their model is designed for method-level fault localization, aiming to provide rankings of faulty methods, which is different from our work.

%also talk about the spectrum and mutation work, also the predicates-based approaches (icse'21 newly work).


%\textbf{CNN-FL~\cite{zhang2019cnn}}; (2) {\bf DeepFL~\cite{DeepFL}} ; and (3) {\bf DeepRL4FL \cite{icse21-fl}}.








%============================= below is from DeepRl4Fl=========================

\iffalse

The Spectrum-based Fault Localization
(SBFL), e.g.,~\cite{Ochiai,abreu2007accuracy,
	jones2005empirical,keller2017critical, liblit2005scalable,
	lucia2014extended,naish2011model, wong2007effective,
	zhang2011localizing}, has been intensively studied in the
literature.  Tarantula \cite{jones2001visualization}, SBI
\cite{liblit2005scalable}, Ochiai \cite{Ochiai} and Jaccard
\cite{abreu2007accuracy}, they share the same basic insight, i.e.,
code elements mainly executed by failed tests are more suspicious.
%Although various SBFL techniques have been proposed, e.g., Tarantula \cite{jones2001visualization}, SBI \cite{liblit2005scalable}, Ochiai \cite{abreu2006evaluation} and Jaccard \cite{abreu2007accuracy}, they share the same basic insight, i.e., code elements mainly executed by failed tests are more suspicious.
%The input of SBFL is the coverage information of all tests and the output is a ranked list of code elements (e.g., statements or methods) according to their descending order of suspiciousness values calculated by specific formula.
The Mutation-based Fault Localization (MBFL), e.g.,~\cite{budd1981mutation,MUSE,musco2017large,zhang2010test,
	zhang2013injecting},
aims to additionally consider mutated code in fault
localization.
%since code elements covered by failed/passed tests may
%not have impact on the corresponding test outcomes.
The examples of MBFL are Metallaxis \cite{papadakis2012using,
	Metallaxis} and MUSE \cite{MUSE}.
%Mutation-based Fault Localization (MBFL), e.g.,~\cite{moon2014ask, zhang2013injecting,budd1981mutation, zhang2010test}, aims to additionally consider impact information for fault localization. Since code elements covered by failed/passed tests may not have any impact on the corresponding test outcomes, e.g., Metallaxis \cite{papadakis2012using, papadakis2015metallaxis} and MUSE \cite{moon2014ask}.
%
%typical MBFL techniques
%use mutation testing \cite{budd1981mutation, zhang2010test, musco2017large} to simulate the impact of each
%code element for more precise fault localization, e.g., Metallaxis \cite{papadakis2012using, papadakis2015metallaxis} and MUSE \cite{moon2014ask}.
%The first general MBFL technique, Metallaxis [24, 26] is based on the
%following intuition: if one mutant has impacts on failed tests (e.g.,
%the tests outcomes change after mutation), its corresponding code
%element may have caused the test failures
\fi
