\subsubsection{\bf RQ2. Impact Analysis Results of Dual-Learning}

\iffalse

\begin{table}[b]
	\caption{RQ2. Results of impact Analysis of multi-tasking framework.}
	{\small
		\begin{center}
			\renewcommand{\arraystretch}{1}
			\begin{tabular}{p{1cm}<{\centering}|p{1.5cm}<{\centering}|p{1.5cm}<{\centering}|p{1.5cm}<{\centering}|p{1.2cm}<{\centering}}
				\hline
				Line$\#$ & Metrics & One Layer Model & Sequentail-based Model & \tool \\
				\hline
				\multirow{3}{*}{1}  & Hit-1@set     & 0.36 & 0.43 & 0.46  \\
									& EXAM          & 0.13 & 0.10 & 0.06  \\
									& EXAM\_AVG     & 0.13 & 0.10 & 0.06  \\
				\hline
				\multirow{4}{*}{2}  & Hit-1@set     & 0.41 & 0.46 & 0.53  \\
									& Hit-2@set     & 0.21 & 0.25 & 0.29  \\
									& EXAM          & 0.13 & 0.12 & 0.08  \\
									& EXAM\_AVG     & 0.29 & 0.22 & 0.15 \\
				\hline
				\multirow{5}{*}{3}  & Hit-1@set     & 0.43 & 0.50 & 0.57 \\
									& Hit-2@set     & 0.21 & 0.23 & 0.28 \\
									& Hit-3@set     & 0.11 & 0.15 & 0.23 \\
									& EXAM          & 0.15 & 0.14 & 0.09 \\
									& EXAM\_AVG     & 0.36 & 0.27 & 0.21 \\
				\hline
				\multirow{6}{*}{4}  & Hit-1@set     & 0.51 & 0.53 & 0.58 \\
									& Hit-2@set     & 0.23 & 0.25 & 0.32 \\
									& Hit-3@set     & 0.08 & 0.09 & 0.15 \\
									& Hit-4@set     & 0.02 & 0.04 & 0.11 \\
									& EXAM          & 0.10 & 0.09 & 0.06 \\
									& EXAM\_AVG     & 0.45 & 0.39 & 0.34 \\
				\hline
				\multirow{7}{*}{5}  & Hit-1@set     & 0.31 & 0.35 & 0.43 \\
									& Hit-2@set     & 0.16 & 0.19 & 0.29 \\
									& Hit-3@set     & 0.09 & 0.13 & 0.17 \\
									& Hit-4@set     & 0.03 & 0.05 & 0.07 \\
									& Hit-5@set     & 0.01 & 0.01 & 0.02 \\
									& EXAM          & 0.14 & 0.13 & 0.08 \\
									& EXAM\_AVG     & 0.48 & 0.44 & 0.39 \\
				\hline
				\multirow{8}{*}{5+}  & Hit-1@set     & 0.29 & 0.32 & 0.37 \\
									& Hit-2@set     & 0.13 & 0.14 & 0.23 \\
									& Hit-3@set     & 0.11 & 0.12 & 0.15 \\
									& Hit-4@set     & 0.03 & 0.05 & 0.12 \\
									& Hit-5@set     & 0.01 & 0.01 & 0.03 \\
									& Hit-5+@set    & 0.01 & 0.01 & 0.01 \\
									& EXAM          & 0.16 & 0.14 & 0.09 \\
									& EXAM\_AVG     & 0.69 & 0.66 & 0.62 \\
				\hline
			\end{tabular}
			
			\label{fig:rq2}
		\end{center}
	}
\end{table}
\fi


Table~\ref{fig:rq2-1} shows the results of different variants of {\tool} for analyzing the impact of Dual-Learning. {\tool} can improve the variants: Only-Statement-Model and Two-Tier Model, on any metrics, indicating that simultaneous dual-learning is effective.

Especially, without the learning of method-level fault localization, Only-Statement-Model has an EXAM score that is doubled than {\tool}. Also without the simultaneous dual learning between method- and statement- level fault localization, the Two-Tier model increases the EXAM score by 41.7\%. The increase of EXAM scores indicates that developers have to spend more effort to locate faulty statements compared with {\tool}.

\begin{table}[t]
	\caption{RQ1.Impact Analysis Results of Dual Learning.}
	{\small
		\begin{center}
			\renewcommand{\arraystretch}{1}
			\begin{tabular}{p{1.35cm}<{\centering}|p{2.4cm}<{\centering}|p{1.7cm}<{\centering}|p{1.2cm}<{\centering}}
				\hline
				Metrics & Only-Statement-Model & Two-Tier Model &  \tool \\			
				\hline
				Hit-1@set   & 304 & 343 & 387 \\
				Hit-2@set	& 111 & 125 & 167\\
				Hit-3@set	& 51 & 61 & 82\\
				Hit-4@set	& 11 & 19 & 46\\
				Hit-5@set	& 3 & 3 & 9\\
				Hit-5+@set	& 2 & 3 & 3\\
				Exam     	& 0.14 & 0.12 & 0.07\\
				\hline
			\end{tabular}
			
			\label{fig:rq2-1}
		\end{center}
	}
\end{table}
