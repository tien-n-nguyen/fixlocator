\subsubsection{\bf RQ3. Sensitivity Analysis Results.}\label{sensi}

Table~\ref{fig:rq3-1} shows the ablation analysis of the
impact of different features on performance. 

\underline{Impact of method-level (ML) features and ML co-change relation.}
Among the method-level features of {\tool}, the {\it method structure}
feature represented as AST has the largest impact. Without the method
structure feature, the performance is decreased by 7.8\% in terms of
Hit-1@Set and increased the EXAM score by 14.3\%. The method content
feature has less impact than the similar-buggy-method one. This
shows that {\em the bugginess nature of a method and similar ones has
more impact than the content of the method itself}.

%The other
%method-level features that have lower impact on performance are in the
%order of \code{Similar Buggy Method} and {\it method content}.

Furthermore, the co-change relations among the methods also have an
impact on the performance of {\tool}. Specifically, without the
co-change feature among methods, the performance is decreased by {\bf
X.X\%} in terms of Hit-1@Set and increased the EXAM score by {\bf
XX.X\%}. This result shows that {\em the novel feature of co-change
relation is important for detecting CC fixing locations}.

%it can decrease the results of Hit-1@Set by xx\% and increase the Exam
%score by xx\%.

\underline{Impact of statement-level (SL) features and SL co-change relation.}
Among the statement-level features, the feature {\it Code Coverage}
has the largest impact. Without {\it Code Coverage} feature, the
performance is decreased by 10.1\% in terms of Hit-1@Set and is
increased by 28.6\% in terms of EXAM score. The AST subtree feature
has more impact on performance than the variable feature.

Moreover, the co-change relations among statements also has an impact
on the performance. Specifically, without co-change relations among
statements, Hit-1@Set is decreased by {\bf XX\%}, and EXAM score is
increased by {\bf XX\%}. Thus, {\em the novel feature of co-change
relation at the statement level is also important}.


\begin{table}[t]
	\caption{RQ3. Sensitivity Analysis of Method- and Statement-Level Features. ML: Method-level; SL: Statement-level.}
	{\small
		\begin{center}
			\renewcommand{\arraystretch}{1}
			\begin{tabular}{p{0.3cm}<{\centering}|p{3cm}|p{0.3cm}<{\centering}|p{0.3cm}<{\centering}|p{0.2cm}<{\centering}|p{0.2cm}<{\centering}|p{0.15cm}<{\centering}|p{0.15cm}<{\centering}|p{0.5cm}<{\centering}}
				\hline
				\multicolumn{2}{c|}{\multirow{2}{*}{Model Variant}}    & \multicolumn{6}{c|}{Hit-n@Set}& \multirow{2}{*}{EXAM}\\
				\cline{3-8}
				                 \multicolumn{2}{c|}{}   &1&2&3&4&5&5+&\\
				
				\hline 
				\multirow{4}{*}{ML}&w/o Method Content              & 366 & 158 & 78  & 39 & 9 & 3   & 0.08\\\cline{2-9}
				&w/o	Method Structure	                        & 357 & 155 & 80  & 40 & 8 & 3   & 0.08\\ \cline{2-9}
				&w/o Similar Buggy Method    	& 361 & 157 & 79  & 44 & 9 & 3   & 0.08\\ \cline{2-9}
				&w/o ML Co-change Relat.         &  &  &   &  &  &    & \\ \cline{2-9}
				\hline
				\multirow{4}{*}{SL}&w/o Code Coverage               & 348 & 151 & 75  & 38 & 7 & 2   & 0.09\\\cline{2-9}
				&w/o	AST Subtree	        & 354 & 153 & 77  & 41 & 8 & 3   & 0.09\\ \cline{2-9}
				&w/o Variables               	& 373 & 162 & 78  & 42 & 9 & 3   & 0.08\\ \cline{2-9}
				&w/o SL Co-change Relat.        &  &  &   &  &  &    & \\ \cline{2-9}
				%&w/o Co-change Relation          & 342 & 148 & 73  & 37 & 6 & 2   & 0.11\\
				\hline
			&	\tool                           & 387 & 167 & 82  & 46 & 9 & 3   & 0.07\\ \cline{2-9}
				\hline
			\end{tabular}
			
			\label{fig:rq3-1}
		\end{center}
	}
\end{table}

\underline{Impact of the depth level of stack trace.} 
As seen in Table~\ref{fig:rq3-2}, the depth level of stack trace can
have a large impact on the performance. Specifically, when the depth =
10, {\tool} can achieve the best performance in all metrics. The cases
with depth = 5 or 15 can bring into analysis too few relevant methods
or too many irrelevant methods. Thus, we chose depth=10 for other
experiments.

%Thus, the results for depth=5 or 15 are lower than the ones of depth=10.

\begin{table}[t]
	\caption{RQ3. Sensitivity Analysis (Key Parameter: Depth of Stack Trace).}
	{\small
		\begin{center}
			\renewcommand{\arraystretch}{1}
			\begin{tabular}{p{1cm}|p{0.3cm}<{\centering}|p{0.3cm}<{\centering}|p{0.3cm}<{\centering}|p{0.3cm}<{\centering}|p{0.3cm}<{\centering}|p{0.3cm}<{\centering}|p{0.7cm}<{\centering}}
				\hline
				\multirow{2}{*}{Depth}    & \multicolumn{6}{c|}{Hit-n@Set}& \multirow{2}{*}{EXAM}\\
				\cline{2-7}
				&1&2&3&4&5&5+&\\
				
				\hline 
				5 			                &  &  &   &  &  &    & \\
				10                          & 387 & 167 & 82  & 46 & 9 & 3   & 0.07\\
				15	                        &  &  &   &  &  &    & \\
				\hline
			\end{tabular}
			
			\label{fig:rq3-2}
		\end{center}
	}
\end{table}


Our results show that all features and components in {\tool} have
impacts on its performance.
