\section{Motivating Example}
\label{motiv:section}

\begin{figure}[t]
	\centering
	\lstset{
		numbers=left,
		numberstyle= \tiny,
		keywordstyle= \color{blue!70},
		commentstyle= \color{red!50!green!50!blue!50},
		frame=shadowbox,
		rulesepcolor= \color{red!20!green!20!blue!20} ,
		xleftmargin=1.5em,xrightmargin=0em, aboveskip=1em,
		framexleftmargin=1.5em,
                numbersep= 5pt,
		language=Java,
    basicstyle=\scriptsize\ttfamily,
    numberstyle=\scriptsize\ttfamily,
    emphstyle=\bfseries,
                moredelim=**[is][\color{red}]{@}{@},
		escapeinside= {(*@}{@*)}
	}
	\begin{lstlisting}[]
public void toSource(final CodeBuilder cb, final int inputSeqNum, final Node root) {
	......
(*@{\color{red}{-	String code = toSource(root, sourceMap);}@*)
(*@{\color{cyan}{+	String code = toSource(root, sourceMap, inputSeqNum == 0);}@*)
	if (!code.isEmpty()) {
		cb.append(code);
	}
	......
}
//--------------------------------------------------------------------------
(*@@Override@*)
String toSource(Node n) {
	initCompilerOptionsIfTesting();
(*@{\color{red}{-	return toSource(n, null);}@*)
(*@{\color{cyan}{+ 	return toSource(n, null, true);}@*)
}
//--------------------------------------------------------------------------
(*@{\color{red}{- private String toSource(Node n, SourceMap sourceMap) {}@*)
(*@{\color{cyan}{+ private String toSource(Node n, SourceMap sourceMap, boolean firstOutput) {}@*)
	......
  builder.setSourceMapDetailLevel(options.sourceMapDetailLevel);
(*@{\color{red}{-   builder.setTagAsStrict(}@*)
(*@{\color{cyan}{+   builder.setTagAsStrict(firstOutput}@*) && 
		options.getLanguageOut(a) == LanguageMode.ECMASCRIPT5_STRICT);
  builder.setLineLengthThreshold(options.lineLengthThreshold);
	......	
}	
	\end{lstlisting}
        \caption{A Multi-statement/Multi-method Bug Fix}
         \label{fig:motiv}
\end{figure}

Let us start with a real-world example on the kind of bug fixes that
require multiple changes to multiple statements of different methods.
Figure~\ref{fig:motiv} shows a bug in Defects4J dataset. The bug
occurred when the method call to \code{setTagAsStrict} did not
consider the first output in its arguments. Therefore, to fix it, a
developer adds a new argument in the method \code{toSource} at line
19, and use that argument in the method call \code{setTagAsStrict
  (firstOutput,...)} at line 23. Because the method \code{toSource} at
line 19 was changed, the two callers at line 3 of the method
\code{toSource} (line 1) and at line 14 of the method \code{toSource}
(line 12) need to be changed accordingly.

%This is a bug from Defects4J data that has three methods need to be fixed at the same time. The bug is caused by doing the method call $setTagAsStrict$ without checking if it is the first output. To fix this bug, firstly, for the method C, we need to add one more parameter $firstOutput$ for the method call $setTagAsStrict$. Then, for the method A and B, because they call the method C or override the method C, when adding a new parameter in the method C, they all need to do the same thing. We need to add $inputSeqNum == 0$ in method A and add $true$ in method B when calling the method C.

%In this bug, there are multiple statements that need to be fixed and they located in three different methods. Also, when analyzing the co-change information, the method A and method C have been fixed together before in the commit history. 
