\section{Feature Extraction}

The first step of \tool is to preprocess the input data into the suitable format and group them into statement and method two levels. So the input of this step is the \tool input, including the java project that needs to do the fault localization with the commit history and the relevant test cases for the project. And the output of this step is two groups of graphs with node features for statement-level and method-level.

Specifically, \tool extract the features from two levels: the method-level and statement-level feature extraction.

\begin{figure}[t]
	\centering
	\includegraphics[width=3.1in]{graphs/step_1_method.png}
	\caption{Method-level Feature Extraction}
	\label{method-level-feature-extraction}
\end{figure}

\subsection{Method-level Feature Extraction}
For the method level, \tool uses the method as the basic unit. And for each method, \tool extracts three key features, including the method content, abstract syntax tree (AST), and the most similar buggy method. 1> For the method content, \tool collect the source code of each method $M$ and link each statement $S_m$ one by one in $M$ as a sequence $Seq_m$ to represent the method content. \tool removes all special characters and uses CamelCase to break down the tokens in the sequence into sub-tokens to reduce the influence of biases. For example, in the Figure \ref{method-level-feature-extraction}, the method content feature in the method $M1$ shows the extracted sequence of sub-tokens $protected void compute ......$. The source code is listed on the input of the Figure \ref{statement-level-feature-extraction}. 2> As for the AST, \tool generates abstract syntax tree $Tree_m$ for each method $M$ by using JDT package \cite{JDT}. The generated AST looks like the example shown on the abstract syntax tree feature in Figure \ref{method-level-feature-extraction}. 3> When extracting the most similar buggy method feature, \tool breaks down the methods into the sequence of sub-tokens just like the method content feature. It then uses GloVe \cite{pennington2014glove} to learn the embedding for each sub-token and replace the sub-tokens with the embedding vectors. After this, \tool calculates the cosine similarity between the current method $m$ and all other buggy methods in the commit history (before the current bug) to find the most similar buggy method $m_b$. For this buggy method, \tool also uses the JDT package to generate the AST for $m_b$. The most similar buggy method feature in Figure \ref{method-level-feature-extraction} shows a possible example for it.

After having these three features for each method, \tool uses three kinds of edges to link them as a graph. First of all, \tool runs test cases for the project. If a test case $t_i$ failed, \tool collect the stack trace for the test case $t_i$. Because the stack trace is sometimes too long, using the crashed position as the root, \tool only picks part of the stack trace $st_i$ with the depth of ten in consideration. It means that on the stack trace $st_i$, there are at most ten methods include $m_1, m_2, ..., m_{10}$ where $m_1$ is the crashed position. For every two method $m_j$ and $m_{j+1}$ among these methods, $m_{j+1}$ calls the $m_j$ in the stack trace. The edge $E_m^s$ direction in it is always from $m_{j+1}$ point to $m_j$. For example, in the Figure \ref{method-level-feature-extraction}, the crashed method is $M1$ and the blue solid edges are the stack trace $st_i$. The figure shows that in the stack trace, $M3$ comes out before $M2$ and $M2$ comes out before $M1$. The second type of edge is the execution path. By using each method $m_j$ in the stack trace $st_i$ as the root method, \tool expands the stack trace $st_i$ by adding the executed methods into the graph. The direction of the execution edges $E_m^e$ is also the same as the call direction. Also, because sometimes the execution path may be very long for a method, we only keep the methods $m_k$ within ten steps from the crashed position $m_1$ that means in the graph, from node $m_k$ to $m_1$, the steps are no more than ten (when counting the steps, \tool ignore the edge direction). The green dotted line in Figure \ref{method-level-feature-extraction} shows an example for this type of edge. With the Figure \ref{method-level-feature-extraction}, the green dotted line reflects the relationship that when running the test case $t_i$ on $M2$, it also executes the method $M4$ based on the method call. And then, it executing the $M3$, it executes the method $M5$, and within $M5$, it also executes method $M6$ based on the method calls inside the methods. The third type of edge is the co-change relation. For this,  \tool collects all commit history from the input java project. If more than one method has changed in one commit, we mark it as a co-change. The co-change contains all the methods that changed together in this commit. To add the co-change relation as one type of edge, \tool makes the co-change relation become a two-directional edge $E_m^c$ (e.g. The orange edge between $M5->M6$ and $M6->M5$ in the Figure \ref{method-level-feature-extraction}).
%
%\begin{itemize}%
%	\item Graph: 
%	\begin{itemize}
%		\item Stack Trace: \tool runs test cases for the project. If a test case $t_i$ failed, \tool collect the stack trace for the test case $t_i$. Because the stack trace may be very bug, by using the crashed position as the root, \tool pick part of the stack trace $st_i$ with the depth of ten. It means that on the stack trace $st_i$, there are ten methods include $m_1, m_2, ..., m_{10}$ where $m_1$ is the crashed position and for every two method $m_j$ and $m_{j+1}$ among them, $m_{j+1}$ calls the $m_j$ in the stack trace. The edge $E_m^s$ direction in it is always from $m_{j+1}$ point to $m_j$ \tool uses $st_i$ as the base graph and the relationship information in $st_i$ is the dynamic information. 
%		\item Execution Path: As for the failed test case $t_i$, \tool also analyzes the execution path. By using each method $m_j$ in the stack trace $st_i$ as the root method, \tool expands the stack trace $st_i$ by adding the executed methods into the graph. The direction of the execution edges $E_m^e$ is also the same as the call direction. Also, because sometimes the execution path may be very long for a method, we only keep the methods $m_k$ within ten steps from the crashed position $m_1$ that means in the graph, from node $m_k$ to $m_1$, the steps are no more than ten (when counting the steps, \tool ignore the edge direction). The added execution information here is the dynamic edge.
%		\item Co-change Information: \tool collects all commit history of the input java project. If more than one java method has changed in one commit, we mark it as a co-change. The co-change contains all the methods that changed together in this commit. Because the co-change does not have the direction, there is one non-directional edge $E^c$ between every two methods in this co-change to represent the co-change relationship. In order to add the co-change relationship into the stack trace, \tool makes the non-directional edge $E^c$ become a two-directional edge $E_m^c$ (e.g.$method_A -> method_B$ and $method_B -> method_A$). This type of edge is the static edge.
%	\end{itemize}
%	\item Node Features: 
%	\begin{itemize}
%		\item Method Content: \tool collect the source code of each method $M$ and link each statement $S_m$ one by one in $M$ as a sequence $Seq_m$ to represent the method content. \tool removes all special characters and uses CamelCase to break down the tokens in the sequence into sub-tokens to reduce the influence of biases. For example, the $setTagAsStrict$ can be break down into $set, Tag, As,$ and $Strict$. The processed sequence of sub-tokens $Seq^p_m$ is used to represent the method content in \tool. This feature is one of the static features that \tool collects from the source code to represent the method.
%		\item Method Structure: \tool generates abstract syntax tree $Tree_m$ for each method $M$ by using JDT package \cite{JDT}. Each tree $Tree_m$ represent the structure of the relevant method $m$. This is one of the static feature that \tool collect from the source code to represent the method.
		%\item Similar Buggy Method: \tool breaks down the methods into the sequence of sub-tokens just like method content feature and then uses GloVe \cite{pennington2014glove} to learn the embedding for each sub-token and replace the sub-tokens with the embedding vectors. After this, \tool calculate the cosine similarity between the current method $m$ and all other buggy methods in the commit history (before current bug) to find the most similar buggy method $m_b$. This is one of the static feature that \tool collect from the source code to represent the method.
%	\end{itemize}	
%\end{itemize} 


\begin{figure}[t]
	\centering
	\includegraphics[width=3.1in]{graphs/step_1_statement.png}
	\caption{Method-level Feature Extraction}
	\label{statement-level-feature-extraction}
\end{figure}

\subsection{Statement-level Feature Extraction}

For the statement level,  \tool uses the statement as the basic unit. And for each statement, \tool extracts three key features, including the code coverage information, sub-AST, and variables to represent the statement. 1> As for the code coverage information, \tool runs the relevant test cases for the input project. For the test case $t_i$, if it passes the statement $s_i$, \tool uses $c_i = 1$ to represent it while if it does not pass the statement $s_i$, \tool uses $c_i = 0$ to represent it. By linking all $c_i$ together, \tool can get $C = <c_1, c_2, ..., c_i>$. Also, \tool uses $r_i = 1$ to represent the the condition $passed$ and uses $r_i = 0$ to represent the condition $failed$ for the test case $t_i$ . By linking all $r_i$ together, \tool can get $R = <r_1, r_2, ..., r_i>$. By concatenate $C$ and $R$, \tool extract the code coverage information feature by $V_{coverage} = <c_1, c_2, ..., c_i, r_1, r_2, ..., r_i>$. The code coverage information feature in Figure \ref{statement-level-feature-extraction} shows an example of it. 2> For the sub-AST, it is very similar to the method level. By using JDT package, \tool can extract the AST for the whole method. And then, \tool searches for the nodes that appears in the statement. By collecting all these nodes and the edges between then, \tool can extract the sub-AST for the statement. 3> For the variables feature, for each statement, \tool collects all the variables $V$ that appeared in it and for each variable $v$ in $V$, \tool uses the $(variable_name variable_type)$ to represent it. Then \tool links all variables $V$ together with $,$ as a sequence $Seq_s$ as one of the static feature that \tool collect from the source code to represent the statement. For example, in Figure \ref{statement-level-feature-extraction}, \tool goes through the statement at $line 5$ in the input method and finds that only one variable $tree$ appears in the statement at $line 5$. So the variable feature here should be like $variable_name variable_type$ where $variable_name$ is $tree$ and $variable_type$ is $BSPTree Euclidean2D$.

With these three features, similar to the method-level, \tool builds three types of edges to link the statements together as a graph. The first type of edge is the program dependency edge (PD). \tool builds the PD by using the tool soot \cite{soot} for the method $m$. For example, in Figure \ref{statement-level-feature-extraction}, the blue edges are the PD edges and they show that in method $m$, statement at $line 4$ controls the statement at $line 5$. And statement at $line 5$ can control the statements at $line 7-8$ and the statements at $line 10-11$. The second type of edge that \tool extracts is the execution flow $E_s^e$. The execution flow is the order that the failed test case $t_i$ went through in the method $m$. For example, the green dotted edges are the execution flow. Within it, we can see that the test case executes $S7-S8$ but did not go through $S10-S11$. Even though in the PD, $S5$ is linked by $S10-S11$. It means that when running the test case $t_i$, it will not pass the $S10$ and $S11$ because of the if checking in $S5$. The last type of edge $E_s^c$ is the co-change relation in the statement level. \tool collects the co-change information for the commit about the statements that changed together before in the current method $m$ and one commit. In Figure \ref{statement-level-feature-extraction}, the commit history shows that $S4$ and $S5$ used to be changed together before. Hence, there is an orange edge that represents the co-change relation between $S4$ and $S5$.


%\begin{itemize}
%	\item Graph: 
%	\begin{itemize}
	%	\item Program Dependency Graph (PDG): \tool builds the PDG by using the tool soot \cite{soot} for the method $m$ that contains the statements that \tool want to analyze. \tool uses the generated PDG as the base graph. Within this graph, there are two types of edges including data dependency and control dependency. Both of these two types of edges are the static edges.
	%	\item Execution Path: \tool collects the execution path of the failed test case $t_i$ within the method $m$ and adds them into the PDG by adding a new type of edge $E_s^e$. The new edge direction is the same as the execution order. This type of edge is the dynamic edge.
%		\item Co-change Information: Similar to the method-level, \tool collects the co-change information for the commit about the statements that changed together before in one commit and the current method $m$. As for adding the co-change information into the PDG, \tool also creates the two-directional edge $E_s^c$ similar to the method level. This type of edge is the static edge.
%	\end{itemize}
%	\item Node Features: 
%	\begin{itemize}
	%	\item Code Coverage Information: \tool runs the relevant test cases for the input java project. For the test case $t_i$, if it passes the statement $s_i$, \tool uses $c_i = 1$ to represent it while if it does not pass the statement $s_i$, \tool uses $c_i = 0$ to represent it. By linking all $c_i$ together as $C = <c_1, c_2, ..., c_i>$, $C$ is considered by \tool as one of the dynamic feature.
	%	\item Statement Structure: Similar to the method-level, \tool generates a sub abstract syntax tree $Tree_s$ for each statement $S$ by using JDT \cite{} package. Each tree $Tree_s$ represent the structure of the relevant statement $s$. This feature is one of the static features that \tool collects from the source code to represent the statement.
	%	\item Variables: For each statement, \tool collects all the variables $V$ that appeared in it and for each variable $v$ in $V$, \tool uses the $(variable_name variable_type)$ to represent it. Then \tool links all variables $V$ together with $,$ as a sequence $Seq_s$ as one of the static feature that \tool collect from the source code to represent the statement. For example, the variables in the statement in line 3 in Figure \ref{fig:motiv} include $root$ and $sourceMap$. \tool generates the feature for them as $root Node, sourceMap SourceMap$ where $root$ and $sourceMap$ are the names and $Node$ and $SourceMap$ are the types.
%	\end{itemize}	
%\end{itemize} 
