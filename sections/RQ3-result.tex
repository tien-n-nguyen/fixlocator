\subsubsection{\bf RQ3. Sensitivity Analysis Results.}

Table~\ref{fig:rq3-1} presents the ablation analysis results of the impact of method- and statement- level features on {\tool}. Our results show that all features have an impact on the {\tool}'s performance, indicating that all designed components can contribute to {\tool}. 

Among the method-level features of {\tool}, the {\it method structure} represented as AST has the largest impact than others on {\tool}. Without the method structure, the performance of {\tool} decreases by 7.8\% in terms of Hit-1@Set and increases the EXAM score by 14.3\%. 
Furthermore, the Co-change relation linking methods also has an impact on the performance of {\tool}. Specifically, it can decrease the results of Hit-1@Set by xx\% and increase the Exam score by xx\%.

Among the statement-level features of {\tool}, the feature {\it Code Coverage} has the largest impact. Without the feature of Code Coverage, the performance of {\tool} decreases 10.1\% in terms of Hit-1@Set and increases the EXAM score by 28.6\%. Moreover, the Co-change relation linking statements has an impact on the performance of {\tool}. Specifically, it can decrease the results of Hit-1@Set by xx\% and increase the Exam score by xx\%.


\begin{table}[t]
	\caption{RQ3. Sensitivity Analysis of Method- and Statement- Level Features. ML: Method-level; SL: Statement-level.}
	{\small
		\begin{center}
			\renewcommand{\arraystretch}{1}
			\begin{tabular}{p{0.3cm}<{\centering}|p{3cm}|p{0.3cm}<{\centering}|p{0.3cm}<{\centering}|p{0.2cm}<{\centering}|p{0.2cm}<{\centering}|p{0.15cm}<{\centering}|p{0.15cm}<{\centering}|p{0.7cm}<{\centering}}
				\hline
				\multicolumn{2}{c|}{\multirow{2}{*}{Approach}}    & \multicolumn{6}{c|}{Hit-n@Set}& \multirow{2}{*}{EXAM}\\
				\cline{3-8}
				                 \multicolumn{2}{c|}{}   &1&2&3&4&5&5+&\\
				
				\hline 
				\multirow{4}{*}{ML}&w/o Method Content              & 366 & 158 & 78  & 39 & 9 & 3   & 0.08\\\cline{2-9}
				&w/o	Method Structure	                        & 357 & 155 & 80  & 40 & 8 & 3   & 0.08\\ \cline{2-9}
				&w/o Similar Buggy Method    	& 361 & 157 & 79  & 44 & 9 & 3   & 0.08\\ \cline{2-9}
				&w/o Co-change Relation          &  &  &   &  &  &    & \\ \cline{2-9}
				\hline
				\multirow{4}{*}{SL}&w/o Code Coverage               & 348 & 151 & 75  & 38 & 7 & 2   & 0.09\\\cline{2-9}
				&w/o	Subtree of AST  	        & 354 & 153 & 77  & 41 & 8 & 3   & 0.09\\ \cline{2-9}
				&w/o Variables               	& 373 & 162 & 78  & 42 & 9 & 3   & 0.08\\ \cline{2-9}
				&w/o Co-change Relation          &  &  &   &  &  &    & \\ \cline{2-9}
				%&w/o Co-change Relation          & 342 & 148 & 73  & 37 & 6 & 2   & 0.11\\
				\hline
			&	\tool                           & 387 & 167 & 82  & 46 & 9 & 3   & 0.07\\ \cline{2-9}
				\hline
			\end{tabular}
			
			\label{fig:rq3-1}
		\end{center}
	}
\end{table}

As seen in Table~\ref{fig:rq3-2}, the depth level of stack trace can have a huge impact on the performance of {\tool}. Specifically, when the depth = 10, {\tool} can achieve the best performance in all metrics. The cases with depth = 5 or 15 can bring in fewer relevant methods or too many irrelevant methods into analysis. Thus, the results for depth=5 or 15 are lower than the ones of depth=10. 

\begin{table}[t]
	\caption{RQ3. Sensitivity Analysis (Key Parameter: Depth of Stack Trace and Execution Path).}
	{\small
		\begin{center}
			\renewcommand{\arraystretch}{1}
			\begin{tabular}{p{1cm}|p{0.3cm}<{\centering}|p{0.3cm}<{\centering}|p{0.3cm}<{\centering}|p{0.3cm}<{\centering}|p{0.3cm}<{\centering}|p{0.3cm}<{\centering}|p{0.7cm}<{\centering}}
				\hline
				\multirow{2}{*}{Depth}    & \multicolumn{6}{c|}{Hit-n@Set}& \multirow{2}{*}{EXAM}\\
				\cline{2-7}
				&1&2&3&4&5&5+&\\
				
				\hline 
				5 			                &  &  &   &  &  &    & \\
				10                          & 387 & 167 & 82  & 46 & 9 & 3   & 0.07\\
				15	                        &  &  &   &  &  &    & \\
				\hline
			\end{tabular}
			
			\label{fig:rq3-2}
		\end{center}
	}
\end{table}