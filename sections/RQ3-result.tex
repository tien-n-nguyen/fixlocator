\subsubsection{\bf RQ3. Sensitivity Analysis Results.}



\begin{table}[t]
	\caption{RQ3. Sensitivity Analysis (Features).}
	{\small
		\begin{center}
			\renewcommand{\arraystretch}{1}
			\begin{tabular}{p{3.3cm}|p{0.3cm}<{\centering}|p{0.3cm}<{\centering}|p{0.3cm}<{\centering}|p{0.3cm}<{\centering}|p{0.3cm}<{\centering}|p{0.3cm}<{\centering}|p{0.7cm}<{\centering}}
				\hline
				\multirow{2}{*}{Approach}    & \multicolumn{6}{c|}{Hit-n@Set}& \multirow{2}{*}{EXAM}\\
				\cline{2-7}
				&1&2&3&4&5&5+&\\
				
				\hline 
				w/o Method Content              & 366 & 158 & 78  & 39 & 9 & 3   & 0.08\\
				w/o	AST	                        & 357 & 155 & 80  & 40 & 8 & 3   & 0.08\\
				w/o Similar Buggy Method    	& 361 & 157 & 79  & 44 & 9 & 3   & 0.08\\
				w/o Code Coverage               & 348 & 151 & 75  & 38 & 7 & 2   & 0.09\\
				w/o	Subtree of AST  	        & 354 & 153 & 77  & 41 & 8 & 3   & 0.09\\
				w/o Variables               	& 373 & 162 & 78  & 42 & 9 & 3   & 0.08\\
				w/o Co-change Relation          & 342 & 148 & 73  & 37 & 6 & 2   & 0.11\\
				\hline
				\tool                           & 387 & 167 & 82  & 46 & 9 & 3   & 0.07\\
				\hline
			\end{tabular}
			
			\label{fig:rq3-1}
		\end{center}
	}
\end{table}

\begin{table}[t]
	\caption{RQ3. Sensitivity Analysis (Key Parameter: Depth of Stack Trace and Execution Path).}
	{\small
		\begin{center}
			\renewcommand{\arraystretch}{1}
			\begin{tabular}{p{1cm}|p{0.3cm}<{\centering}|p{0.3cm}<{\centering}|p{0.3cm}<{\centering}|p{0.3cm}<{\centering}|p{0.3cm}<{\centering}|p{0.3cm}<{\centering}|p{0.7cm}<{\centering}}
				\hline
				\multirow{2}{*}{Depth}    & \multicolumn{6}{c|}{Hit-n@Set}& \multirow{2}{*}{EXAM}\\
				\cline{2-7}
				&1&2&3&4&5&5+&\\
				
				\hline 
				5 			                &  &  &   &  &  &    & \\
				10                          & 387 & 167 & 82  & 46 & 9 & 3   & 0.07\\
				15	                        &  &  &   &  &  &    & \\
				\hline
			\end{tabular}
			
			\label{fig:rq3-2}
		\end{center}
	}
\end{table}