\subsection{\bf RQ3. Sensitivity Analysis Results}~\label{sensi}

%(Table~\ref{fig:rq3-1}

\begin{table}[t]
	\caption{RQ3. Sensitivity Analysis of Method- and Statement-Level Features. ML: Method-level; SL: Statement-level}
        \vspace{-6pt}
	{\small
		\begin{center}
			\renewcommand{\arraystretch}{1}
			\begin{tabular}{p{0.3cm}<{\centering}|p{3cm}|p{0.3cm}<{\centering}|p{0.3cm}<{\centering}|p{0.2cm}<{\centering}|p{0.2cm}<{\centering}|p{0.15cm}<{\centering}|p{0.15cm}<{\centering}}
				\hline
				\multicolumn{2}{c|}{\multirow{2}{*}{Model Variant}}    & \multicolumn{6}{c}{Hit-N}\\
				\cline{3-8}
				                 \multicolumn{2}{c|}{}   &1&2&3&4&5&5+\\
				
				\hline 
				\multirow{4}{*}{ML}&w/o Method Content              & 366 & 158 & 78  & 39 & 9 & 3   \\\cline{2-8}
				&w/o	Method Structure	                        & 357 & 155 & 80  & 40 & 8 & 3   \\ \cline{2-8}
				&w/o Similar Buggy Method    	& 361 & 157 & 79  & 44 & 9 & 3   \\ \cline{2-8}
				&w/o ML Co-change Rel.         & 355 & 152 & 77  & 40 & 8 &  3   \\ \cline{2-8}
				\hline
				\multirow{4}{*}{SL}&w/o Code Coverage               & 348 & 151 & 75  & 38 & 7 & 2   \\\cline{2-8}
				&w/o	AST Subtree	        & 354 & 153 & 77  & 41 & 8 & 3   \\ \cline{2-8}
				&w/o Variables               	& 373 & 162 & 78  & 42 & 9 & 3   \\ \cline{2-8}
				&w/o SL Co-change Relation       & 351 & 150 & 76 & 39 & 7 &  2   \\ \cline{2-8}
				\hline
			&	{\tool}                           & 387 & 167 & 82  & 46 & 9 & 3  \\ \cline{2-8}
				\hline
			\end{tabular}
			
			\label{fig:rq3-1}
		\end{center}
	}
\end{table}

%\begin{table}[t]
%	\caption{RQ3. Sensitivity Analysis of Method- and Statement-Level Features. ML: Method-level; SL: Statement-level.}
%        \vspace{-6pt}
%	{\footnotesize
%		\begin{center}
%			\renewcommand{\arraystretch}{1}
%			\begin{tabular}{p{0.3cm}<{\centering}|p{3cm}|p{0.3cm}<{\centering}|p{0.3cm}<{\centering}|p{0.2cm}<{\centering}|p{0.2cm}<{\centering}|p{0.15cm}<{\centering}|p{0.15cm}<{\centering}|p{0.5cm}<{\centering}}
%				\hline
%				\multicolumn{2}{c|}{\multirow{2}{*}{Model Variant}}    & \multicolumn{6}{c|}{Hit-n@Set}& \multirow{2}{*}{EXAM}\\
%				\cline{3-8}
%				                 \multicolumn{2}{c|}{}   &1&2&3&4&5&5+&\\
%				
%				\hline 
%				\multirow{4}{*}{ML}&w/o Method Content              & 366 & 158 & 78  & 39 & 9 & 3   & 0.08\\\cline{2-9}
%				&w/o	Method Structure	                        & 357 & 155 & 80  & 40 & 8 & 3   & 0.08\\ \cline{2-9}
%				&w/o Similar Buggy Method    	& 361 & 157 & 79  & 44 & 9 & 3   & 0.08\\ \cline{2-9}
%				&w/o ML Co-change Relat.         & 355 & 152 & 77  & 40 & 8 &  3  &0.09 \\ \cline{2-9}
%				\hline
%				\multirow{4}{*}{SL}&w/o Code Coverage               & 348 & 151 & 75  & 38 & 7 & 2   & 0.09\\\cline{2-9}
%				&w/o	AST Subtree	        & 354 & 153 & 77  & 41 & 8 & 3   & 0.09\\ \cline{2-9}
%				&w/o Variables               	& 373 & 162 & 78  & 42 & 9 & 3   & 0.08\\ \cline{2-9}
%				&w/o SL Co-change Relat.        & 351 & 150 & 76 & 39 & 7 &  2  & 0.11 \\ \cline{2-9}
%				\hline
%			&	\tool                           & 387 & 167 & 82  & 46 & 9 & 3   & 0.07\\ \cline{2-9}
%				\hline
%			\end{tabular}
%			
%			\label{fig:rq3-1}
%		\end{center}
%	}
%\end{table}

%Table~\ref{fig:rq3-1} shows the ablation analysis for
%different method-/statement-level features.
%of the impact of different features and relations among
%methods/statements on {\tool}'s performance.

%\underline{Impact of method-level (ML) features and ML co-change relation.}

\subsubsection{{\bf Impact of method-level (ML) features and ML co-change relation}}

Among the method-level features/attributes of {\tool}, the {\it method
structure feature represented as AST has larger impact}. Without
the method structure feature, Hit-1 is decreased by 7.8\%.
%and EXAM score is increased by 14.3\%.
The method content feature has less impact than the
similar-faulty-method one. This shows that {\em the bugginess nature
of a method and similar ones has more impact than the tokens of the
method itself}.

Moreover, the {\em feature of co-change relations among methods
has} 
%ML co-change relations among methods have a
{\em the larger impact} than any ML features.
%on the performance.
Specifically, without the co-change feature among methods, Hit-1 is
decreased by 8.3\%.
%and EXAM score is increased by 28.6\%.
%This result shows that {\em our novel feature of co-change relation is
%important for detecting CC fixing locations}.

%it can decrease the results of Hit-1@Set by xx\% and increase the Exam
%score by xx\%.

%Tien

%\underline{Impact of statement-level (SL) features and SL co-change relation.}

\subsubsection{\bf Impact of statement-level (SL) features and SL co-change relation}

Among the statement-level features, {\it Code Coverage
has the largest impact}. Without {\it Code Coverage} feature,
Hit-1 is decreased by 10.1\%.
%and EXAM score is increased by 28.6\%.
%the performance is decreased by 10.1\% in terms of Hit-1@Set and is
%increased by 28.6\% in terms of EXAM score.
%The subtree feature has more impact than the variable feature.
%
The co-change relations among statements also have larger
impact than any SL features/attributes. Specifically, without
co-change relations among statements, Hit-1 is decreased by 9.3\%.

%and EXAM score is increased by 57.1\%. Thus, {\em the novel feature of
%co-change relation is also very important for detecting CC fixing
%statements}.




%\underline{Impact of the depth level of stack trace.} 

\subsubsection{\bf Impact of the depth level of stack trace}

In Table~\ref{fig:rq3-2},
%the depth level of stack trace can have a large impact on the
%performance. Specifically,
when depth=10, {\tool} can achieve the best performance. The
cases with depth= 5 or 15 can bring into analysis too few or too many
irrelevant methods. Thus, we chose depth=10 for other experiments.

%Thus, the results for depth=5 or 15 are lower than the ones of depth=10.

\begin{table}[t]
	\caption{RQ3. Sensitivity Analysis (Depth of Traces)}
        \vspace{-9pt}
	{\small
		\begin{center}
			\renewcommand{\arraystretch}{1}
			\begin{tabular}{p{1cm}|p{0.3cm}<{\centering}|p{0.3cm}<{\centering}|p{0.3cm}<{\centering}|p{0.3cm}<{\centering}|p{0.3cm}<{\centering}|p{0.3cm}<{\centering}}
				\hline
				\multirow{2}{*}{Depth}    & \multicolumn{6}{c}{Hit-N}\\
				\cline{2-7}
				&1&2&3&4&5&5+\\
				
				\hline 
				5 			                & 371 & 162 & 74  & 42 & 8 & 3 \\
			{\bf	10}                         & 387 & 167 & 82  & 46 & 9 & 3   \\
				15	                        & 368 & 158 & 71  & 39 & 7 & 3 \\
				\hline
			\end{tabular}
			
			\label{fig:rq3-2}
		\end{center}
	}
\end{table}

\subsubsection{{\bf Illustrating Example}}
\label{sec:example}

\begin{figure}[t]
	\centering
	\lstset{
		numbers=left,
		numberstyle= \tiny,
		keywordstyle= \color{blue!70},
		commentstyle= \color{red!50!green!50!blue!50},
		frame=shadowbox,
		rulesepcolor= \color{red!20!green!20!blue!20} ,
		xleftmargin=1.5em,xrightmargin=0em, aboveskip=1em,
		framexleftmargin=1.5em,
		numbersep= 5pt,
		language=Java,
		basicstyle=\scriptsize\ttfamily,
		numberstyle=\scriptsize\ttfamily,
		emphstyle=\bfseries,
		moredelim=**[is][\color{red}]{@}{@},
		escapeinside= {(*@}{@*)}
	}
	\begin{lstlisting}[]
public UnivariateRealPointValuePair optimize(final FUNC f, GoalType goal, double min, double max) throws FunctionEvaluationException {
(*@{\color{red}{  - return optimize(f, goal, min, max, 0);@*)
(*@{\color{cyan}{  + return optimize(f, goal, min, max, min + 0.5 * (max - min));@*)
}
public UnivariateRealPointValuePair optimize(final FUNC f, GoalType goal, double min, double max, double startValue) throws Func...Exception {
       ...
       try {
(*@{\color{red}{-        final double bound1 = (i == 0) ? min : min + generator.nextDouble() * (max - min);@*)
(*@{\color{red}{-        final double bound2 = (i == 0) ? max : min + generator.nextDouble() * (max - min);@*)
(*@{\color{red}{-        optima[i] = optimizer.optimize(f, goal, FastMath.min(bound1, bound2), FastMath.max(bound1, bound2));@*)
(*@{\color{cyan}{+       final double s = (i == 0) ? startValue : min + generator.nextDouble() * (max - min);@*)
(*@{\color{cyan}{+       optima[i] = optimizer.optimize(f, goal, min, max, s);@*)
   ...
}
\end{lstlisting}
        \vspace{-15pt}
	\caption{An Illustrating Example}
	\label{example}
\end{figure}

Figure~\ref{example} shows an example in our dataset that {\tool}
correctly identified all {\em four CC fixing statements} in its predicted
set (lines 2, 8, 9, and 10 in two methods) that need to be fixed
together (CC fixing locations).

The statement-only model detects only line 2 as faulty. It completely
missed lines 8--10 of the \code{optimize} method (line 5). In contrast,
the cascading model detects lines 8--10, however, its \code{MethFL}
considers the first method (\code{optimize(...)} at line 1) as
non-faulty, thus, it did not detect the buggy line 2 due
to its cascading architecture.

% Table generated by Excel2LaTeX from sheet 'Sheet1'
\begin{table}[t]
  \centering
  \caption{Ranking of CC Fixing Locations for Figure~\ref{example}}
  \vspace{-9pt}
  {\footnotesize
    \begin{tabular}{|lcccc|}
    \toprule
     {\textbf{Locs}} & \multicolumn{1}{l} {\textbf{CNN-FL}} & \multicolumn{1}{l}{\textbf{DeepFL}} & \multicolumn{1}{l}{\textbf{DeepRL4FL}} & \multicolumn{1}{l|}{\textbf{FixLocator}} \\
    \midrule
    Line 2 & {\bf 1}     & 22    & {\bf 2}     & {\bf 1} \\
    \midrule
    Line 8 & 24    & {\bf 3}     & 6     & {\bf 3} \\
    \midrule
    Line 9 & 25    & {\bf 4}     & 7     & {\bf 4} \\
    \midrule
    Line 10 & 50+    & 13    & 16    & {\bf 2} \\
    \bottomrule
    \end{tabular}%
  \label{tab:ranking}%
  }
\end{table}%

Table~\ref{tab:ranking} displays the ranking from the models for the
faulty statements in the example. {\tool} correctly predicted {\em the set
of those four CC fixing statements} and ranked them at the top 4
positions. CCN-FL, DeepFL, and DeepRL4FL detect only 1, 2, and 1
faulty statements (bold cells) in their top-4 resulting lists,
respectively. The top-4 resulting list from CNN-FL is [2, X, X, X] (X
is the line number in other clean methods not shown in
Figure~\ref{example}); the list from DeepFL is [X, X, 8, 9]; and that
from DeepRL4FL is [X, 2, X, 169 of \code{optimize}]. In brief, the
baselines are not designed to detect CC fixing locations, thus,
their top-$K$ lists are not~correct.



%Our results show that all features and components in {\tool} have
%impacts on its performance.
